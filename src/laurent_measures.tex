\subsection{Laurent Measures}

\begin{definition}
  \label{laurent-measures}
  \lean{laurent_measures}
  Let $0 < r' < 1$ be a real number, and let $S$ be a profinite set.
  Then $\Lm_{r'}(S)$ denotes TODO
\end{definition}

For every $\alpha\in\R$ let $\lfloor \alpha\rfloor\in\Z$ be the largest integer that satisfies $\lfloor \alpha\rfloor\le \alpha$: in particular,
\[
  \alpha-1 < \lfloor \alpha\rfloor\le \alpha
  \qquad\text{and}\qquad
  \alpha\geq 0 \Longrightarrow \lfloor \alpha\rfloor \geq 0.
\]

Now fix $0< \xi<1$ and let $x\in\R_{\geq 0}$. 

\begin{definition}
  \label{theta.y}
  \lean{theta.y}
  \leanok
	For all $n\in\N$, set
	\[
		y_n(x)=y_n=
		\begin{cases}
			x&\text{ if } n = 0 \\
			y_{n-1} - \left\lfloor \frac{y_{n-1}}{\xi^{n-1}} \right\rfloor \xi^{n-1} & \text{ if } n \ge 1
		\end{cases}
	\]
\end{definition}

\begin{lemma}
  \label{theta.finite_sum}
  \uses{theta.y}
  \lean{theta.finite_sum}
  \leanok
  For all $n\in\N$, we have
  \[
    y_{n+1} = x - \sum_{i=0}^n \Big\lfloor \frac{y_i}{\xi^i} \Big\rfloor \xi^i.
  \]
\end{lemma}

\begin{proof}
  \leanok
  By induction: when $n=0$ this boils down to
  \[
    y_1=x-\left\lfloor \frac{y_0}{\xi^0}\right\rfloor \xi^0
  \]
  which is the definition of $y_1$ since $x=y_0$.

	The step $n\to (n+1)$ goes as follows:
	\[
		y_{n+1} \overset{\text{def}}{=} y_{n} - \left\lfloor \frac{y_{n}}{\xi^{n}} \right\rfloor \xi^{n}
		\overset{\text{Ind.}}{=} \left(x - \sum_{i=0}^{n-1} \Big\lfloor \frac{y_i}{\xi^i} \Big\rfloor\xi^i\right) -
		\left\lfloor \frac{y_{n}}{\xi^{n}} \right\rfloor\xi^{n}
		= x - \sum_{i=0}^n \Big\lfloor \frac{y_i}{\xi^i} \Big\rfloor \xi^i.
		\qedhere
	\]
\end{proof}

\begin{lemma}
  \label{theta.bounded}
  \uses{theta.y}
	The sequence
	\[
		n \mapsto \left\lfloor \frac{y_n}{\xi^n} \right\rfloor
	\]
	is bounded.
\end{lemma}

\begin{proof}
	By definition, for all $n\geq 1$,
	\[
		\frac{y_n}{\xi^n} = \frac{1}{\xi^n} \left(y_{n-1} - \left\lfloor \frac{y_{n-1}}{\xi^{n-1}} \right\rfloor \xi^{n-1}\right)
    = \frac{1}{\xi} \left(\frac{y_{n-1}}{\xi^{n-1}} - \left\lfloor \frac{y_{n-1}}{\xi^{n-1}} \right\rfloor\right)
    \le \frac{1}{\xi}\cdot 1
    = \frac{1}{\xi}
  \]
  Thus, the sequence $n\mapsto \frac{y_n}{\xi^n}$ is bounded by $\max\{x=y_0,\xi^{-1}\}$
  and the same holds, \emph{a fortiori}, for the sequence $n\mapsto \lfloor\frac{y_n}{\xi^n}\rfloor$. 
\end{proof}

\begin{lemma}
  \label{theta.limit_y}
  \uses{theta.y}
  \lean{theta.limit_y}
  \leanok
  We have
  \[
    \lim_{n\to+\infty} y_n =0.
  \]
\end{lemma}

\begin{proof}
  \uses{theta.bounded}
  \leanok
  Indeed, for every $n\in \N$,
  \[
    \frac{y_n}{\xi^n} - 1 <\left\lfloor{\frac{y_n}{\xi^n}}\right\rfloor \le \frac{y_n}{\xi^n}
  \]
  so, multiplying everything by $\xi^n > 0$,
  \[
    y_n - \xi^n <\left\lfloor{\frac{y_n}{\xi^n}}\right\rfloor\xi^n \le y_n.
  \]
  Taking limits,
  \[
    \lim_{n\to+\infty} y_n - \lim_{n\to+\infty} \xi^n
    = \lim_{n\to+\infty} y_n
    \le \lim_{n\to+\infty} \left\lfloor \frac{y_n}{\xi^n} \right\rfloor \xi^n
    = 0\le\lim_{n\to+\infty} y_n.
  \]
  By Lemma~\ref{theta.bounded},
  the term $\left\lfloor\frac{y_n}{\xi^n}\right\rfloor\xi^n$
  tends to $0$ as $n\to +\infty$ because $0<\lvert \xi\rvert< 1$.
  The lemma follows.
\end{proof}

\begin{proposition}
  \label{theta.has_sum_x}
  \uses{theta.y}
  \lean{theta.has_sum_x,theta.summable_floor}
  \leanok
  %For all $i\in\N$, let
  %\[
  %a_i=\left\lfloor\frac{y_i}{\xi^i}\right\rfloor \in \N.
  %\]
  %The sequence $i\mapsto a_i\in \N$, satisfies
  \begin{enumerate}
    \item For all $0<r <1$,
      \[
        \sum_{i=0}^{+\infty} \left\lfloor\frac{y_i}{\xi^i}\right\rfloor r^i \qquad\text{converges}.
      \]
    \item
      \[
        \sum_{i=0}^{+\infty} \left\lfloor\frac{y_i}{\xi^i}\right\rfloor\xi^i = x;
      \]
  \end{enumerate}
\end{proposition}

\begin{proof} 
  \uses{theta.bounded,theta.finite_sum,theta.limit_y}
  \leanok
  The first point follows from Lemma~\ref{theta.bounded}, using that $0<r<1$.

  For the second point, taking limits in Lemma~\ref{theta.finite_sum}, we obtain
  \[
    \lim_{n\to +\infty}y_{n+1}=x-\sum_{i=0}^{+\infty}a_i\xi^i.
  \]
  The result follows from Lemma~\ref{theta.limit_y}.
\end{proof}

The next theorem is the first statement of Theorem~6.9 in~\cite{Analytic}.

\begin{definition}
  \label{theta}
  \lean{theta.ϑ}
  \leanok
  Fix $0<\xi <1$, and let $\theta_{\xi}\colon \Z((T))_r\rightarrow \R$ be the evaluation map
  \[
    \sum a_nT^n\longmapsto \sum a_n(\xi)T^n.
  \]
\end{definition}

\begin{theorem}
  \label{theta.surjective}
  \uses{theta}
  \lean{theta.ϑ_surjective}
  \leanok
  The map $\theta_{\xi}$ is surjective.
\end{theorem}

\begin{proof}
  \uses{theta.has_sum_x}
  \leanok
  Pick $x\in \R$, and consider the power series
  \[
    F(T)=\sum_{n\geq 0}\left\lfloor \frac{y_n(x)}{\xi^m}\right\rfloor T^n\in\Z[\![T]\!]
  \]
  where $\{y_n(x)\}$ is the sequence defined in Definition~\ref{theta.y}.
  By Proposition~\ref{theta.has_sum_x}, the series converges on the open unit disks,
  so it belongs to $\Z((T))_r$ and it specializes to $x$ when evaluated at $T=\xi$.
\end{proof}

% vim: ts=2 et sw=2 sts=2
