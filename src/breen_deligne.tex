\section{Breen--Deligne data}

The goal of this section is to a give a precise statement of the Breen--Deligne resolution.
We first give the statement, and provide details later.

\begin{theoremx}[Breen--Deligne]
  \label{BD_orig}
  For an abelian group $A$, there is a resolution, functorial in~$A$, of the form
  \[
    \ldots \to \bigoplus_{i=1}^{n_i} \mathbb Z[A^{r_{ij}}] \to \ldots
    \to \mathbb Z[A^3] \oplus \mathbb Z[A^2] \to \mathbb Z[A^2] \to \mathbb Z[A] \to A \to 0.
  \]
\end{theoremx}

I (Johan Commelin) have not figured out the details.
But it seems to be possible to avoid the $\bigoplus_{i=1}^{n_i}$,
so we will aim for something like the following statement.

\begin{theoremx}
  \label{BD_reso}
  For an abelian group $A$, there is a resolution, functorial in~$A$, of the form
  \[
    \ldots \to \mathbb Z[A^{n_{i}}] \to \ldots \to \mathbb Z[A^2] \to \mathbb Z[A] \to A \to 0.
  \]
\end{theoremx}

What does a homomorphism $f \colon \mathbb Z[A^m] \to \mathbb Z[A^n]$
that is functorial in~$A$ look like? We should perhaps say more precisely
what we mean by this. The idea is that $m$ and $n$ are fixed, and
for each abelian group $A$ we have a group homomorphism
$f_A\colon \mathbb Z[A^m] \to \mathbb Z[A^n]$ such that if $\phi:A\to B$
is a group homomorphism inducing $\phi_i:\Z[A^i]\to\Z[B^i]$ for each
natural number $i$ then the obvious square commutes: $\phi_n \circ f_A = f_B \circ \phi_m$.

The map $f_A$ is specified by what it does to the generators
$(a_1, a_2, a_3, \dots, a_m)\in A^m$. It can send such an element
to an arbitrary element of $\mathbb Z[A^n]$, but one can check
that universality implies that $f_A$ will be a $\mathbb Z$-linear combination of
``basic universal maps'', where a ``basic universal map'' is one that
sends $(a_1, a_2, \dots, a_m)$ to $(t_1, \dots, t_n)$,
where $t_i$ is a $\mathbb Z$-linear combination $c_{i,1} \cdot a_1 + \dots + c_{i,m} \cdot a_m$.
So a ``basic universal map'' is specified by the $n \times m$-matrix $c$.

\begin{definition}
  \label{basic_universal_map}
  \lean{breen_deligne.basic_universal_map}
  \leanok
  A \emph{basic universal map} from exponent $m$ to $n$,
  is an $n \times m$-matrix with coefficients in~$\mathbb Z$.
\end{definition}

\begin{definition}
  \label{universal_map}
  \lean{breen_deligne.universal_map}
  \leanok
  \uses{basic_universal_map}
  A \emph{universal map} from exponent $m$ to $n$,
  is a formal $\mathbb Z$-linear combination of basic universal maps from exponent $m$ to $n$.
\end{definition}

We point out that basic universal maps can be composed by matrix multiplication,
and this formally induces a composition of universal maps. As mentioned above, one can also check (this has been formalised in Lean) that this construction gives a bijection between universal maps from exponent $m$ to $n$ and functorial collections $f_A:\Z[A^m]\to\Z[A^n]$.

\begin{definition}
  \label{sigma_add}
  \lean{breen_deligne.σ_add}
  \leanok
  \uses{universal_map}
  The addition on $A^n$ induces a universal map
  $\sigma_\alpha \colon \mathbb Z[(A^n)^2] \to \mathbb Z[A^n]$,
  namely the formal generator $(I_n I_n)$, where $I_n$ denotes the $n \times n$ identity matrix.
  (Here $\alpha$ stands for ``addition''.)
\end{definition}

\begin{definition}
  \label{sigma_proj}
  \lean{breen_deligne.σ_proj}
  \leanok
  \uses{universal_map}
  The formal sum of the two projections $(A^n)^2 \to A^n$
  induces a universal map $\sigma_\pi \colon \mathbb Z[(A^n)^2] \to \mathbb Z[A^n]$,
  namely the formal sum $(I_n 0) + (0 I_n)$,
  where $I_n$ denotes the $n \times n$ identity matrix, and $0$ the $n \times n$ zero matrix.
  (Here $\pi$ stands for ``projections''.)
\end{definition}

\begin{definition}
  \label{BD_double}
  \lean{breen_deligne.universal_map.double}
  \leanok
  \uses{universal_map}
  Let $f$ be a universal map from exponent~$m$ to~$n$.
  Then $f \oplus f$ denotes the universal map from exponent~$2m$ to~$2n$,
  that applies $f$ componentwise.
  If $f$ is a generator (i.e.\ a basic universal map)
  then $f \oplus f$ is
  \[
    \begin{pmatrix}
    f & 0 \\
    0 & f
    \end{pmatrix}.
  \]
\end{definition}

\begin{definition}
  \label{BD_data}
  \lean{breen_deligne.data}
  \lean{breen_deligne.is_complex}
  \leanok
  \uses{sigma_add, sigma_proj}
  A tuple $(n, f)$ of \emph{Breen--Deligne data}
  consists of a sequence of exponents $n_0, n_1, n_2, \dots \in \mathbb N$,
  and universal maps $f_i$ from exponent $n_{i+1}$ to $n_i$.

  Such a tuples is a \emph{complex} if for all $i$ we have $f_i \circ f_{i+1} = 0$.

  A \emph{universal morphism} of Breen--Deligne data (or complexes) $(m,f)\to (n,g)$ is a collection of universal maps $\phi_i$ from exponent $m_i$ to $n_i$ such that $g_i\circ \phi_{i+1}=\phi_i\circ f_i$ as universal maps from exponent $m_{i+1}$ to $n_i$ (i.e., the squares commute). 
\end{definition}

\begin{definition}
  \label{BD_data_double}
  \lean{breen_deligne.data.double}
  \leanok
  \uses{BD_data, BD_double}
  If $(n, f)$ is a tuple of Breen--Deligne data,
  then $(n, f) \oplus (n, f)$ is the tuple
  consisting of exponents $2n_i$ and universal maps $f_i \oplus f_i$.
\end{definition}

The two universal map $\sigma_\alpha$ and~$\sigma_\pi$ explained in the examples above, can be checked to induce universal maps of complexes: $(n,f) \oplus (n,f) \to (n,f)$.

\begin{definition}
  \label{BD_homotopy}
  \lean{breen_deligne.homotopy}
  \leanok
  \uses{BD_data, BD_double}
  A \emph{homotopy} for a tuple $(n, f)$ of Breen--Deligne data
  is a homotopy between the maps of complexes
  \[
    \sigma_\alpha, \sigma_\pi \colon (n,f) \oplus (n,f) \to (n,f)
  \]
  In other words, it consists of universal maps $h_i$ from exponent $2n_i$ to $n_{i+1}$,
  such that $f_0\circ h_0=\sigma_\alpha-\sigma_\pi$ as universal maps from exponent $2n_0$ to $n_0$, and for all $i\geq0$ we have
  \[
    f_{i+1} \circ h_{i+1} + h_i \circ (f_i \oplus f_i) = \sigma_\alpha - \sigma_\pi
    \]
    as universal maps from exponent $2n_{i+1}$ to $n_{i+1}$. Note that the first condition is morally the $i=-1$ case of the displayed equation, if we set $h_{-1}=0$.
\end{definition}

\begin{definition}
  \label{BD_package}
  \lean{breen_deligne.package}
  \leanok
  \uses{BD_data, BD_homotopy}
  A \emph{Breen--Deligne package}
  is a triple $(n, f, h)$,
  such that $(n, f)$ is Breen--Deligne data that is a complex,
  and $h$ is a homotopy for $(n,f)$.
\end{definition}

\begin{definition}
  \label{BD_eg}
  \lean{breen_deligne.eg}
  \leanok
  \uses{BD_package}
  We will now construct an example of a Breen--Deligne package.
  In some sense, it is the ``easiest'' solution to the conditions posed above.
  The exponents will be $n_i = 2^i$, and the homotopies $h_i$ will be the identity.
  Under these constraints, we recursively construct the universal maps $f_i$:
  \[
    f_0 = \sigma_\alpha - \sigma_\pi,
    \quad
    f_{i+1} = (\sigma_\alpha - \sigma_\pi) - (f_i \oplus f_i).
  \]
  We leave it as exercise for the reader, to verify that
  with these definitions $(n, f, h)$ forms a Breen--Deligne package.
\end{definition}

We now make three definitions that will make precise
some conditions between constants that will be needed
when we construct Breen--Deligne complexes of normed abelian groups.

\begin{definition}
  \label{basic_suitable}
  \lean{breen_deligne.basic_universal_map.suitable}
  \leanok
  \uses{basic_universal_map}
  Let $f$ be a basic universal map from exponent~$m$ to~$n$.
  Let $c_1, c_2 \in \mathbb R_{\ge 0}$.
  We say that $(c_1, c_2)$ is \emph{$f$-suitable}, if for all $i$
  \[
    \sum_j c_1|f_{ij}| \le c_2.
  \]
\end{definition}

To orient the reader: later on we will be considering maps on normed abelian groups induced from universal maps,
and this inequality will guarantee that if $\|m\|\leq c_1$ then $\|f(m)\|\leq c_2$.

\begin{definition}
  \label{universal_suitable}
  \lean{breen_deligne.universal_map.suitable}
  \leanok
  \uses{universal_map, basic_suitable}
  Let $f$ be a universal map from exponent~$m$ to~$n$.
  Let $c_1, c_2 \in \mathbb R_{\ge 0}$.
  We say that $(c_1, c_2)$ is \emph{$f$-suitable}, if for all basic universal maps $g$
  that occur in the formal sum $f$,
  the pair of nonnegative reals $(c_1, c_2)$ is $g$-suitable.
\end{definition}

\begin{definition}
  \label{BD_suitable}
  \lean{breen_deligne.package.suitable}
  \leanok
  \uses{BD_package, universal_suitable}
  Let $(n, f, h)$ be a Breen--Deligne package,
  and let $c = (c_0, c_1, \dots)$ be a sequence of nonnegative real numbers.
  We say that $c$ is \emph{$(n,f,h)$-suitable},
  if for all $i$, the pair $(c_{i+1}, c_i)$ is $f_i$-suitable.

  (Note! The order $(c_{i+1}, c_i)$ is contravariant
  compared to Definition~\ref{universal_suitable}.
  This is because of the contravariance of $\hat V(\_)$;
  see Definition~\ref{eval_CLCFPTinv}.)
\end{definition}

% vim: ts=2 et sw=2 sts=2
