\section{Completions of locally constant functions}

\begin{definition}
  \label{Vhat}
  \lean{NormedGroup.LCC}
  \leanok
  Let $V$ be a normed abelian group, and $X$ a compact topological space.
  We denote by $V(X)$ the normed abelian group of locally constant functions $X \to V$
  with respect to the sup norm.
  With $\hat V(X)$ we denote the completion of $V(X)$.

  These constructions are functorial in bounded group homomorphisms $V \to V'$
  and contravariantly functorial in continuous maps $X \to X'$.
\end{definition}

We continue to use the notation of before:
let $r' > 0, c \ge 0$ be real numbers,
and let $M$ be a profinitely filtered pseudo-normed group with $r'$-action by $T^{-1}$.

\begin{definition}
  \label{basic_eval_CLCFP}
  \lean{breen_deligne.basic_universal_map.eval_CLCFP}
  \leanok
  \uses{basic_eval_FP, Vhat}
  Let $f$ be a basic universal map from exponent~$m$ to~$n$,
  and let $(c_1, c_2)$ be $f$-suitable.
  We get an induced map
  \[
    \hat V(f) \colon \hat V(M_{\le c_1}^m) \to \hat V(M_{\le c_2}^n).
  \]
\end{definition}

\begin{definition}
  \label{eval_CLCFP}
  \lean{breen_deligne.universal_map.eval_CLCFP}
  \leanok
  \uses{universal_suitable, Vhat_basic_eval_Mbar}
  Let $f$ be a universal map from exponent~$m$ to~$n$,
  and let $(c_1, c_2)$ be $f$-suitable.
  We get an induced map
  \[
    \hat V(f) \colon \hat V(M_{\le c_1}^m) \to \hat V(M_{\le c_2}^n)
  \]
  that is the sum $\sum n_g V(g)$,
  if $f$ is the formal sum $\sum n_g g$
  of basic universal maps.
\end{definition}

\begin{definition}
  \label{normed_with_aut}
  \lean{normed_with_aut}
  \leanok
  Let $r > 0$ be a real number.
  An \emph{$r$-normed $\mathbb Z[T^{\pm 1}]$-module}
  is a normed abelian group $V$
  endowed with an automorphism $T \colon V \to V$ such that
  for all $v \in V$ we have $\|T(v)\| = r\|v\|$.
\end{definition}

\begin{lemma}
  \label{Vhat_normed_with_aut}
  \lean{NormedGroup.normed_with_aut_LCC}
  \leanok
  \uses{Vhat, normed_with_aut}
  Let $r \in \mathbb R_{\ge 0}$,
  and let $V$ be an $r$-normed $\mathbb Z[T^{\pm 1}]$-module.
  Let $X$ be a compact space.
  Then $\hat V(X)$ is naturally an $r$-normed $\mathbb Z[T^{\pm 1}]$-module,
  with the action of $T$ given by post-composition.
\end{lemma}

\begin{proof}
  \leanok
  Formalised, but omitted from this text.
\end{proof}

Let $r > 0$, and let $V$ be an $r$-normed $\mathbb Z[T^{\pm 1}]$-module.
Assume $r' \le 1$.

\begin{definition}
  \label{CLCFPTinv}
  \lean{CLCFPTinv}
  \leanok
  \uses{Mbar_with_Tinv, Vhat_normed_with_aut}
  There are two natural actions of $T^{-1}$ on $\hat V(M_{\le c})$.
  The first comes from the $r'$-action of $T^{-1}$ on $M$
  which gives a continuous map
  \[
    M_{\le cr'} \to M_{\le c}
  \]
  and thus a map
  \[
    (T^{-1})^* \colon \hat V(M_{\le c}) \to \hat V(M_{\le cr'}).
  \]
  The other comes from Lemma~\ref{Vhat_normed_with_aut},
  using the $r$-normed $\mathbb Z[T^{\pm 1}]$-module $V$.
  We get a map
  \[
    [T^{-1}] \colon \hat V(M_{\le c}) \to \hat V(M_{\le c}),
  \]
  that we can compose with the map
  $\hat V(M_{\le c}) \to \hat V(M_{\le cr'})$,
  obtained from the natural inclusion $M_{\le cr'} \to M_{\le c}$.
  We thus end up with two maps
  \[
    (T^{-1})^*, [T^{-1}] \colon \hat V(M_{\le c}) \to \hat V(M_{\le cr'}).
  \]
  and we define $\hat V(M_{\le c})^{T^{-1}}$
  to be the equalizer of $(T^{-1})^*$ and $[T^{-1}]$.
  In other words, the kernel of $(T^{-1})^* - [T^{-1}]$.
\end{definition}

\begin{definition}
  \label{eval_CLCFPTinv}
  \lean{breen_deligne.universal_map.eval_CLCFPTinv}
  \leanok
  \uses{CLCFPTinv, eval_CLCFP}
  Let $f$ be a universal map from exponent~$m$ to~$n$,
  and let $(c_1, c_2)$ be $f$-suitable.

  The natural map from Definition~\ref{eval_CLCFP}
  restricts to a map
  \[
    \hat V(f)^{T^{-1}} \colon \hat V(M_{\le c_2}^n)^{T^{-1}} \to \hat V(M_{\le c_1}^m)^{T^{-1}}
  \]
\end{definition}

