\documentclass[11pt]{amsart}
\usepackage{amsfonts, amsthm, amssymb, amsmath, stmaryrd, color, enumerate, relsize}
\usepackage[pdftex]{graphicx}
\usepackage{mathrsfs,array}
\usepackage{xy}
\usepackage{hyperref}
\usepackage{tikz}
\usetikzlibrary{snakes}
\input xy
\xyoption{all}


\setlength{\textwidth}{6.5in}
\setlength{\oddsidemargin}{-0.1in}
\setlength{\evensidemargin}{-0.1in}
\def\from{\colon}
\def\GG{\mathscr{G}}
\DeclareMathOperator{\Spa}{Spa}
\DeclareMathOperator{\Sha}{Sha}
\DeclareMathOperator{\Space}{Space}
\DeclareMathOperator{\Sp}{Sp}
\DeclareMathOperator{\Sh}{Sh}
%\DeclareMathOperator{\sp}{sp}
\DeclareMathOperator{\Spf}{Spf}
\DeclareMathOperator{\Spm}{Spm}
\DeclareMathOperator{\Spd}{Spd}
\DeclareMathOperator{\Spv}{Spv}
\DeclareMathOperator{\MaxSpec}{MaxSpec}
\DeclareMathOperator{\Proj}{Proj}
\DeclareMathOperator{\Cont}{Cont}
\newcommand{\cycl}{{\mathrm{cycl}}}
\newcommand{\ad}{{\mathrm{ad}}}
\DeclareMathOperator{\rec}{rec}
\DeclareMathOperator{\nr}{nr}
\DeclareMathOperator{\GL}{GL}
\DeclareMathOperator{\Gal}{Gal}
\DeclareMathOperator{\Hom}{Hom}
\DeclareMathOperator{\Ext}{Ext}
\newcommand{\intHom}{\underline{\Hom}}
\DeclareMathOperator{\Aut}{Aut}
\DeclareMathOperator{\Mod}{Mod}
\DeclareMathOperator{\VB}{Bun}
\DeclareMathOperator{\Isoc}{Isoc}
\DeclareMathOperator{\Rep}{Rep}
\DeclareMathOperator{\tr}{tr}
\DeclareMathOperator{\ch}{char}
\DeclareMathOperator{\Frob}{Frob}
\DeclareMathOperator{\Spec}{Spec}
\DeclareMathOperator{\Bun}{Bun}
\DeclareMathOperator{\Sht}{Sht}
\DeclareMathOperator{\Sets}{Sets}
\DeclareMathOperator{\Perf}{Perf}
\DeclareMathOperator{\Perfd}{Perfd}
\DeclareMathOperator{\Frac}{Frac}
\DeclareMathOperator{\CAff}{CAff}
\newcommand{\AnRing}{{\mathrm{AnRing}}}
\DeclareMathOperator{\AnSpec}{AnSpec}
\DeclareMathOperator{\Adic}{Adic}
\DeclareMathOperator{\qcqs}{qcqs}
\DeclareMathOperator{\op}{op}
\DeclareMathOperator{\Lie}{Lie}
\DeclareMathOperator{\Coeq}{Coeq}
\DeclareMathOperator{\Newt}{Newt}
\DeclareMathOperator{\val}{val}
\newcommand{\crys}{{\mathrm{crys}}}
\newcommand{\FF}{{\mathrm{FF}}}
\DeclareMathOperator{\Gr}{Gr}
\DeclareMathOperator{\Res}{Res}
\DeclareMathOperator{\coker}{coker}
\DeclareMathOperator{\Fil}{Fil}
\DeclareMathOperator{\gr}{gr}
\DeclareMathOperator{\diag}{diag}
\DeclareMathOperator{\Grass}{Grass}
\DeclareMathOperator{\adm}{adm}
\DeclareMathOperator{\Def}{Def}
\DeclareMathOperator{\GM}{GM}
\DeclareMathOperator{\PGL}{PGL}
\DeclareMathOperator{\BD}{BD}
\DeclareMathOperator{\Cond}{Cond}
\DeclareMathOperator{\Ab}{Ab}
\DeclareMathOperator{\Ch}{Ch}
\DeclareMathOperator{\Dem}{Dem}
\DeclareMathOperator{\Latt}{Latt}
\DeclareMathOperator{\Fun}{Fun}
\DeclareMathOperator{\LCA}{LCA}
\DeclareMathOperator{\ProFin}{ProFin}
\DeclareMathOperator{\Pro}{Pro}
\DeclareMathOperator{\AnSp}{AnSp}
\DeclareMathOperator{\Fin}{Fin}
\DeclareMathOperator{\inv}{inv}
\DeclareMathOperator{\colim}{colim}
\newcommand{\ind}{{\mathrm{ind}}}
\def\OO{\mathcal{O}}
\renewcommand{\op}{{\mathrm{op}}}
\newcommand{\loc}{{\mathrm{loc}}}
\newcommand{\Dr}{{\mathrm{Dr}}}
\renewcommand{\inf}{{\mathrm{inf}}}
\newcommand{\cons}{{\mathrm{cons}}}
\newcommand{\red}{{\mathrm{red}}}
\newcommand{\perf}{{\mathrm{perf}}}
\newcommand{\et}{{\mathrm{\acute{e}t}}}
\newcommand{\fet}{{\mathrm{f\acute{e}t}}}
\newcommand{\proet}{{\mathrm{pro\acute{e}t}}}
\newcommand{\qproet}{{\mathrm{qpro\acute{e}t}}}
\newcommand{\BC}{{\mathcal B\mathcal C}}
\newcommand{\heart}{\heartsuit}
\def\F{\mathbf{F}}
\def\Z{\mathbf{Z}}
\def\R{\mathbf{R}}
\def\C{\mathbf{C}}
\def\D{\mathbf{D}}
\def\E{\mathcal{E}}
\def\PP{\mathbf{P}}
\def\Q{\mathbf{Q}}
\def\A{\mathbf{A}}
\def\G{\mathbf{G}}
\def\Gm{\mathbf{G}_{\text{m}}}
\def\Ga{\mathbf{G}_{\text{a}}}
\def\X{\mathcal{X}}
\def\Y{\mathcal{Y}}
\def\injects{\hookrightarrow}
\def\congto{\stackrel{\sim}{\to}}
\newcommand{\dR}{{\mathrm{dR}}}
\DeclareMathOperator{\inte}{int}
\DeclareMathOperator{\rank}{rank}
\DeclareMathOperator{\pr}{pr}
\DeclareMathOperator{\LT}{LT}
\newcommand{\Fl}{{\mathscr F\!\ell}}
\newcommand{\solid}{{\mathsmaller{\blacksquare}}}
\DeclareMathOperator{\Solid}{Solid}
\newcommand{\preanalytic}{\triangleright}
\newcommand{\CondMod}{\mathrm{Mod}^{\mathrm{cond}}}
\renewcommand*{\diamond}{\diamondsuit}
\renewcommand*{\hat}{\widehat}
\renewcommand*{\tilde}{\widetilde}
\renewcommand{\det}{\mathrm{det}}
%\DeclareMathOperator{\inf}{inf} Already defined
\newcommand{\Drinfeld}{Drinfeld}


\newcommand{\set}[1]{\left\{ #1 \right\}}
\newcommand{\abs}[1]{\left| #1 \right|}
\newcommand{\tatealgebra}[1]{\langle #1 \rangle}
\newcommand{\powerseries}[1]{[\![ #1 ]\!]}
\newcommand{\laurentseries}[1]{(\!( #1 )\!)}



\numberwithin{equation}{section}
\newtheorem{theorem}{Theorem}
\numberwithin{theorem}{section}
\newtheorem{lemma}[theorem]{Lemma}
\newtheorem{corollary}[theorem]{Corollary}
\newtheorem{proposition}[theorem]{Proposition}
\newtheorem{conjecture}[theorem]{Conjecture}
\newtheorem{problem}[theorem]{Problem}
\newtheorem{assumption}[theorem]{Assumption}
\newtheorem{defprop}[theorem]{Definition/Proposition}



\theoremstyle{definition}
\newtheorem{remark}[theorem]{Remark}
\newtheorem{exercise}[theorem]{Exercise}
\newtheorem{warning}[theorem]{Warning}
\newtheorem{definition}[theorem]{Definition}
\newtheorem{question}[theorem]{Question}
\newtheorem{example}[theorem]{Example}
\newtheorem{examples}[theorem]{Examples}
\newtheorem{observation}[theorem]{Observation}

% new enumerate and itemize environments with different spacing conventions
\newenvironment{altenumerate}
   {\begin{list}
      {\textup{(\theenumi)} }
      {\usecounter{enumi}
       \setlength{\labelwidth}{0pt}
       \setlength{\labelsep}{2pt}
       \setlength{\leftmargin}{0pt}
       \setlength{\itemsep}{\the\smallskipamount}
       \renewcommand{\theenumi}{\roman{enumi}}
      }}
   {\end{list}}
\newenvironment{altitemize}
   {\begin{list}
      {$\bullet$ }
      {\setlength{\labelwidth}{0pt}
       \setlength{\labelsep}{2pt}
       \setlength{\leftmargin}{0pt}
       \setlength{\itemsep}{\the\smallskipamount}
      }}
   {\end{list}}

%\renewcommand{\familydefault}{cm}
\date{\today}

\setcounter{tocdepth}{1}

\title{Blueprint for the Liquid Tensor Experiment}

\author{Johan Commelin (all results by Peter Scholze and Dustin Clausen)}

\begin{document}

\maketitle

\begin{remark}
	This text is based on the lecture notes on Analytic Geometry~\cite{Analytic},
	by Peter Scholze.
\end{remark}

\section{Lecture IX: End of proof}

% Recall that in the last lecture, we reduced Theorem~\ref{thm:key} to the following result.

% \begin{theorem}\label{thm:explicit1} Fix radii $1>r'>r>0$. Then for all $r$-normed $\mathbb Z[T^{\pm 1}]$-modules $V$ and all profinite sets $S$, the map
% \[
% R\Hom_{\mathbb Z[T^{-1}]}(\mathcal M(S,\mathbb Z((T))_{r'}),\widehat{V})\to \widehat{V}(S)
% \]
% is a quasi-isomorphism.
% \end{theorem}

% We note that actually the case of $\widehat{V}=\mathbb Z((T))_r^B$ itself seems to carry all essential difficulty: We believe that any argument one can give in that case will work in general.

% In that sense, we have simplified the target of the $R\Hom$ as far as possible. It is time to understand the source. In the last lecture, we observed that the theorem automatically implies a more precise version bounding the norms of preimages of differentials. Our proof will actually go back to such explicit bounds, but for explicit resolutions of the source, which we will now construct. Write
% \[
% \overline{\mathcal M}_{r'}(S) := \mathcal M(S,\mathbb Z((T))_{r'})/\mathbb Z[T^{-1}][S].
% \]
% This is, as a condensed abelian group (but not as $\mathbb Z[T^{-1}]$-module), a direct summand $\mathcal M(S,T\mathbb Z[[T]]_{r'})$ of $\mathcal M(S,\mathbb Z((T))_{r'})$, allowing only positive powers of $T$.\footnote{At this point, we critically use that we chose $\mathbb Z[T^{-1}]$ as our base ring; already taking $\mathbb Z[T^{\pm 1}]$ would destroy this argument.} To see this, use Proposition~\ref{prop:freecondensedgroup} to see that the contribution to $\mathcal M(S,\mathbb Z((T))_{r'})$ from nonpositive powers of $T$ is exactly $\mathbb Z[T^{-1}][S]$. In particular, we can write
% \[
% \overline{\mathcal M}_{r'}(S) = \bigcup_{c>0} \overline{\mathcal M}_{r'}(S)_{\leq c},
% \]
% where
% \[
% \overline{\mathcal M}_{r'}(S)_{\leq c} = \varprojlim_i \overline{\mathcal M}_{r'}(S_i)_{\leq c}
% \]
% when writing $S$ as an inverse limit of finite sets $S_i$, and for finite $S$
% \[
% \overline{\mathcal M}_{r'}(S)_{\leq c} = \{(\sum_{n\geq 1} a_{n,s} T^n)_s\mid \sum_{n\geq 1,s\in S} |a_{n,s}|r^n\leq c\}.
% \]

% We need to resolve this explicitly as a condensed $\mathbb Z[T^{-1}]$-module. The Breen-Deligne resolution gives us the resolution
% \[
% \ldots\to \mathbb Z[\overline{\mathcal M}_{r'}(S)^2]\to \mathbb Z[\overline{\mathcal M}_{r'}(S)]\to \overline{\mathcal M}_{r'}(S)\to 0
% \]
% as condensed $\mathbb Z[T^{-1}]$-modules; as in the appendix, we assume (for notational convenience) that each term is of the form $\mathbb Z[\overline{\mathcal M}_{r'}(S)^a]$ (instead of some finite direct sum of such). Moreover, each term $\mathbb Z[\overline{\mathcal M}_{r'}(S)^a]$ admits a two-term resolution
% \[
% 0\to \mathbb Z[T^{-1}][\overline{\mathcal M}_{r'}(S)^a]\xrightarrow{T^{-1}-[T^{-1}]} \mathbb Z[T^{-1}][\overline{\mathcal M}_{r'}(S)^a]\to \mathbb Z[\overline{\mathcal M}_{r'}(S)^a]\to 0
% \]
% that is functorial in the $\mathbb Z[T^{-1}]$-module $\mathbb Z[\overline{\mathcal M}_{r'}(S)^a]$. This gives a resolution of $\overline{\mathcal M}_{r'}(S)$ by the double complex
% \[\xymatrix{
% \ldots\ar[r] &\mathbb Z[T^{-1}][\overline{\mathcal M}_{r'}(S)^2]\ar[d]^{T^{-1}-[T^{-1}]}\ar[r] & \mathbb Z[T^{-1}][\overline{\mathcal M}_{r'}(S)]\ar[d]^{T^{-1}-[T^{-1}]}\\
% \ldots\ar[r] &\mathbb Z[T^{-1}][\overline{\mathcal M}_{r'}(S)^2]\ar[r] & \mathbb Z[T^{-1}][\overline{\mathcal M}_{r'}(S)].
% }\]
% Unfortunately, the terms are not of the form $\mathbb Z[T^{-1}][E_i]$ with profinite $E_i$. To remedy this, we write the double complex as the filtered colimit of the double complexes
% \begin{equation}\label{eq:doublecomplex}
% \xymatrix{
% \ldots\ar[r] &\mathbb Z[T^{-1}][\overline{\mathcal M}_{r'}(S)^2_{\leq r'c_1c}]\ar[d]^{T^{-1}-[T^{-1}]}\ar[r] & \mathbb Z[T^{-1}][\overline{\mathcal M}_{r'}(S)_{\leq r'c}]\ar[d]^{T^{-1}-[T^{-1}]}\\
% \ldots\ar[r] &\mathbb Z[T^{-1}][\overline{\mathcal M}_{r'}(S)^2_{\leq c_1c}]\ar[r] & \mathbb Z[T^{-1}][\overline{\mathcal M}_{r'}(S)_{\leq c}]
% }\end{equation}
% for varying $c>0$, where the constants $c_1,\ldots>0$ are fixed and chosen so that the transition maps are well-defined, cf.~Lemma~\ref{lem:constantsdeligne} in the appendix to this lecture.

% It is easy to understand what happens to the vertical part under mapping to $V$:

% \begin{lemma}\label{lem:Tinv} For any $r$-normed $\mathbb Z[T^{\pm 1}]$-module $V$, any $c>0$ and any $a$, the map
% \[
% \widehat{V}(\overline{\mathcal M}_{r'}(S)_{\leq c}^a)\xrightarrow{T^{-1}-[T^{-1}]^\ast} \widehat{V}(\overline{\mathcal M}_{r'}(S)_{\leq r'c}^a)
% \]
% is surjective, has norm bounded by $r^{-1}+1$, and for any $f\in \widehat{V}(\overline{\mathcal M}_{r'}(S)_{\leq r'c}^a)$ and $\epsilon>0$ there is some $g\in \widehat{V}(\overline{\mathcal M}_{r'}(S)_{\leq c}^a)$ with $f(x)=T^{-1}g(x)-g(T^{-1}x)$ and $||g||\leq \frac{r}{1-r}(1+\epsilon) ||f||$.
% \end{lemma}

% \begin{proof} Given $f: \overline{\mathcal M}_{r'}(S)_{\leq r'c}^a\to \widehat{V}$, choose an extension to a map $\tilde{f}: \overline{\mathcal M}_{r'}(S)^a\to \widehat{V}$ with $||\tilde{f}||\leq (1+\epsilon)||f||$. Such an extension exists: By induction (and using a sequence of $\epsilon_n$'s with $\prod_n (1+\epsilon_n)\leq 1+\epsilon$), it suffices to see that for any closed immersion $A\subset B$ of profinite sets and a map $f_A: A\to \widehat{V}$, there is an extension $f_B: B\to \widehat{V}$ of $f_A$ with $||f_B||\leq (1+\epsilon)||f_A||$. To see this, write $f_A$ as a (fast) convergent sum of maps that factor over a finite quotient of $A$; for maps factoring over a finite quotient of $A$, the extension is clear (and can be done in a norm-preserving way), as any map from $A$ to a finite set can be extended to a map from $B$ to the same finite set.

% Given $\tilde{f}$, we can now define $g: \overline{\mathcal M}_{r'}(S)_{\leq c}^a$ by
% \[
% g(x) = T\tilde{f}(x)+T^2\tilde{f}(T^{-1}x)+\ldots+T^{n+1}\tilde{f}(T^{-n}x)+\ldots\in \widehat{V};
% \]
% then $||g||\leq \frac r{1-r}||\tilde{f}||\leq \frac r{1-r}(1+\epsilon)||f||$.
% \end{proof}

% Let $\widehat{V}(\overline{\mathcal M}(S)_{\leq c})^{T^{-1}}\subset \widehat{V}(\overline{\mathcal M}(S)_{\leq c})$ be the kernel of this map, with the subspace norm. For varying $c$, we now get varying normed complexes. We will need to use the following qualitative notion of exactness.

\begin{definition} For each sufficiently large $c$ (i.e.~all $c\geq c_0$ for some $c_0>0$), let
\[
C_c^\bullet: C_c^0\to C_c^1\to\ldots
\]
be a complex of complete normed abelian groups, and for $c'>c$, let $\mathrm{res}_{c',c}^i: C_{c'}^\bullet\to C_c^\bullet$ be a map of complexes, satisfying the obvious associativity condition. This datum is admissible if all differentials and maps $\mathrm{res}_{c',c}^i$ are norm-nonincreasing.

For integers $m\geq 0$ and constants $k>0$, $c_0'>0$, the datum $(C_c^\bullet)_c$ is $\leq k$-exact in degrees $\leq m$ and for $c\geq c_0'$ if the following condition is satisfied. For all $c\geq c_0'$ and all $x\in C_{kc}^i$ with $i\leq m$ there is some $y\in C_c^{i-1}$ (which is defined to be $0$ when $i=0$) such that
\[
||\mathrm{res}_{kc,c}^i(x)-d_c^{i-1}(y)||_{C_c^i}\leq k||d_{kc}^i(x)||_{C_{kc}^{i+1}}.
\]
\end{definition}

We apply this to
\[
C_c^\bullet: \widehat{V}(\overline{\mathcal M}_{r'}(S)_{\leq c})^{T^{-1}}\to \widehat{V}(\overline{\mathcal M}_{r'}(S)_{\leq c_1c}^2)^{T^{-1}}\to \ldots
\]
given by mapping \eqref{eq:doublecomplex} into $\hat{V}$ and using Lemma~\ref{lem:Tinv}.
Now we state the following result, which is our main goal.

\textbf{N.b.:}
It differs from Theorem~9.4 of \cite{Analytic} only in one aspect:
we assume that the sets $S$ are finite, rather than profinite.

\begin{theorem}\label{thm:explicit2}
	Fix radii $1>r'>r>0$. For any $m$ there is some $k$ and $c_0$ such that for all finite sets $S$ and all $r$-normed $\mathbb Z[T^{\pm 1}]$-modules $V$, the system of complexes
\[
C_c^\bullet: \widehat{V}(\overline{\mathcal M}_{r'}(S)_{\leq c})^{T^{-1}}\to \widehat{V}(\overline{\mathcal M}_{r'}(S)_{\leq c_1c}^2)^{T^{-1}}\to \ldots
\]
is $\leq k$-exact in degrees $\leq m$ for $c\geq c_0$.
\end{theorem}

% Let us first check that this implies Theorem~\ref{thm:explicit1}.

% \begin{proof}[Theorem~\ref{thm:explicit2} implies Theorem~\ref{thm:explicit1}] By the preceding discussion, one can compute
% \[
% R\Hom_{\mathbb Z[T^{-1}]}(\overline{\mathcal M}_{r'}(S),\widehat{V})
% \]
% as the derived inverse limit of $C_c^\bullet$ over all $c>0$; equivalently, all $c\geq c_0$. Theorem~\ref{thm:explicit2} implies that for any $m\geq 0$ the pro-system of cohomology groups $H^m(C_c^\bullet)$ is pro-zero (as $H^m(C_{kc}^\bullet)\to H^m(C_c^\bullet)$ is zero). Thus, the derived inverse limit vanishes, as desired.
% \end{proof}

% We remark that Theorem~\ref{thm:explicit2} reduces formally to the case that $S$ is finite; we make this reduction.

We will prove Theorem~\ref{thm:explicit2} by induction on $m$. Unfortunately, the induction requires us to prove a stronger statement. This goes as follows. Consider any polyhedral lattice $\Lambda$, by which we mean a finite free abelian group equipped with a norm $||\cdot||_\Lambda: \Lambda\otimes \mathbb R\to \mathbb R$ (so $\Lambda\otimes \mathbb R$ is a Banach space) that is given by the supremum of finitely many linear functions on $\Lambda$ with rational coefficients; equivalently, the ``unit ball'' $\{\lambda\in \Lambda\otimes \mathbb R\mid ||\lambda||_\Lambda\leq 1\}$ is a rational polyhedron.

Endow $\Hom(\Lambda,\overline{\mathcal M}_{r'}(S))$ with the subspaces
\[
\Hom(\Lambda,\overline{\mathcal M}_{r'}(S))_{\leq c} = \{f: \Lambda\to \overline{\mathcal M}_{r'}(S)\mid \forall x\in \Lambda, f(x)\in \overline{\mathcal M}_{r'}(S)_{\leq c||x||}\}.
\]
As $\Lambda$ is polyhedral, it is enough to check the given condition for finitely many $x$.

We can then define double complexes like \eqref{eq:doublecomplex}. Lemma~\ref{lem:Tinv} stays true with the same constants. Now we claim the following generalization of Theorem~\ref{thm:explicit2}.

\begin{theorem}\label{thm:explicit3} Fix radii $1>r'>r>0$. For any $m$ there is some $k$ such that for all polyhedral lattices $\Lambda$ there is a constant $c_0(\Lambda)>0$ such that for all finite sets $S$ and all $r$-normed $\mathbb Z[T^{\pm 1}]$-modules $V$, the system of complexes
\[
C_{\Lambda,c}^\bullet: \widehat{V}(\Hom(\Lambda,\overline{\mathcal M}_{r'}(S))_{\leq c})^{T^{-1}}\to \widehat{V}(\Hom(\Lambda,\overline{\mathcal M}_{r'}(S))_{\leq c_1c}^2)^{T^{-1}}\to \ldots
\]
is $\leq k$-exact in degrees $\leq m$ for $c\geq c_0(\Lambda)$.
\end{theorem}

We note that the constants $c_1,c_2,\ldots$ implicit in the choice of the complex are chosen once and for all (after fixing $r$ and $r'$), and it can be ensured that the transition maps in the complex are norm-nonincreasing. Indeed, with the $c_i$ chosen as in Lemma~\ref{lem:constantsdeligne}, the maps
\[
\widehat{V}(\Hom(\Lambda,\overline{\mathcal M}_{r'}(S))_{\leq c_ic}^{a_i})\to \widehat{V}(\Hom(\Lambda,\overline{\mathcal M}_{r'}(S))_{\leq c_{i+1}c}^{a_{i+1}})
\]
will have bounded norm, independently of $V$ (as they are a certain universal finite sum of maps induced by maps between the profinite sets in paranthesis, each of which induces a map of norm bounded by $1$), so on the subspace of $T^{-1}$-invariants, one can shrink the norm down to $1$ by shrinking $c_{i+1}$. We make and fix this choice of the $c_i$ for the statement of Theorem~\ref{thm:explicit3}, and the rest of the proof.

\begin{proposition}\label{prop:key} Fix an integer $m\geq 0$ and a constant $k$. Then there exists an $\epsilon>0$ and a constant $k_0$, depending (only) on $k$ and $m$, with the following property.

Consider an admissible system of double complexes $M^{p,q}_c$, $p,q\geq 0$, $c\geq c_0$, of complete normed abelian groups as well as some $k'\geq k_0$ and some $H>0$, such that
\[\xymatrix{
M^{0,0}_c\ar[r]^{d'^{0,0}_c}\ar[d]^{d^{0,0}_c} & M^{0,1}_c\ar[r]^{d'^{0,1}_c}\ar[d]^{d^{0,1}_c} & M^{0,2}_c\ar[r]^{d'^{0,2}_c}\ar[d]^{d^{0,2}_c} & \ldots\\
M^{1,0}_c\ar[r]^{d'^{1,0}_c}\ar[d]^{d^{1,0}_c} & M^{1,1}_c\ar[r]^{d'^{1,1}_c}\ar[d]^{d^{1,1}_c} & M^{1,2}_c\ar[r]^{d'^{1,2}_c}\ar[d]^{d^{1,2}_c} & \ldots\\
M^{2,0}_c\ar[r]^{d'^{2,0}_c}\ar[d]^{d^{2,0}_c} & M^{2,1}_c\ar[r]^{d'^{2,1}_c}\ar[d]^{d^{2,1}_c} & \ddots\\
\vdots & \vdots
}\]
\begin{enumerate}
\item for $j=0,\ldots,m$, the columns $M^{p,j}_c$ are $\leq k$-exact in degrees $\leq m$ for $c\geq c_0$;
\item for $i=0,\ldots,m+1$, the rows $M^{i,q}_c$ are $\leq k$-exact in degrees $\leq m-1$ for $c\geq c_0$;
\item for $q=0,\ldots,m$ and $c\geq c_0$, there is a map $h^q_{k'c}: M^{0,q+1}_{k'c}\to M^{1,q}_c$ with
\[
||h^q_{k'c}(x)||_{M^{1,q}_c}\leq H||x||_{M^{0,q+1}_{k'c}}
\]
for all $x\in M^{0,q+1}_c$, and such that for all $c\geq c_0$ and $x\in M^{0,q}_{k'^2c}$, one has
\begin{equation}\label{eq:homotopicmapsmall}
||\mathrm{res}_{k'^2c,k'c}^{1,q}(d^{0,q}(x))\pm h^q_{k'^2c}(d'^{0,q}_{k'^2c}(x))\pm d'^{1,q-1}_{k'c}(h^{q-1}_{k'^2c}(x))||_{M^{1,q}_{k'c}}\leq \epsilon ||\mathrm{res}_{k'^2c,c}^{0,q}(x)||_{M^{0,q}_c}.
\end{equation}
\end{enumerate}
Then the first row is $\leq \max(k'^2,2k_0H)$-exact in degrees $\leq m$ for $c\geq c_0$.
\end{proposition}

We note that the bound on the homotopy is of a peculiar nature, in that the bound only depends on a deep restriction of $x$.

\begin{proof} First, we treat the case $m=0$. If $m=0$, we claim that one can take $\epsilon=\tfrac 1{2k}$ and $k_0=k$. We have to prove exactness at the first step. Let $x_{k'^2c}\in M^{0,0}_{k'^2c}$ and denote $x_{k'c}=\mathrm{res}_{k'^2c,k'c}^{0,0}(x)$ and $x_c=\mathrm{res}_{k'^2c,c}^{0,0}(x)$. Then by assumption (1) (and $k'\geq k$), we have
\[
||x_c||_{M^{0,0}_c}\leq k||d^{0,0}_{k'c}(x_{k'c})||_{M^{1,0}_{k'c}}.
\]
On the other hand, by (3),
\[
||\mathrm{res}_{k'^2c,k'c}^{1,0}(d^{0,0}_{k'^2c}(x))\pm h^0_{k'^2c}(d'^{0,0}_{k'^2c}(x))||_{M^{1,0}_{k'c}}\leq \epsilon ||x_c||_{M^{0,0}_c}.
\]
In particular, noting that $\mathrm{res}_{k'^2c,k'c}^{1,0}(d^{0,0}_{k'^2c}(x)) = d^{0,0}_{k'c}(x_{k'c})$, we get
\[
||x_c||_{M^{0,0}_c}\leq k||d^{0,0}_{k'c}(x_{k'c})||_{M^{1,0}_{k'c}}\leq k\epsilon ||x_c||_{M^{0,0}_c} + kH ||d'^{0,0}_{k'^2c}(x)||_{M^{0,1}_{k'^2c}}.
\]
Thus, taking $\epsilon=\tfrac 1{2k}$ as promised, this implies
\[
||x_c||_{M^{0,0}_c}\leq 2kH ||d'^{0,0}_{k'^2c}(x)||_{M^{0,1}_{k'^2c}}.
\]
This gives the desired $\leq \max(k'^2,2k_0H)$-exactness in degrees $\leq m$ for $c\geq c_0$.

Now we argue by induction on $m$. Consider the complex $N^{p,q}$ given by $M^{p,q+1}$ for $q\geq 1$ and $N^{p,0} = M^{p,1}/\overline{M^{p,0}}$ (the quotient by the closure of the image, which is also the completion of $M^{p,1}/M^{p,0}$), equipped with the quotient norm. Using the normed version of the snake lemma, Proposition~\ref{prop:snakelemma} in the appendix to this lecture, one checks that this satisfies the assumptions for $m-1$, with $k$ replaced by $\max(k^4,k^3+k+1)$.
\end{proof}

Finally, we can prove the key combinatorial lemma, ensuring that any element of $\Hom(\Lambda,\overline{\mathcal M}_{r'}(S))$ can be decomposed into $N$ elements whose norm is roughly $\tfrac 1N$ of the original element. As preparation, we have the following simple result.

\begin{lemma} Let $\Lambda$ be a finite free abelian group, let $N$ be a positive integer, and let $\lambda_1,\ldots,\lambda_m\in \Lambda$ be elements. Then there is a finite subset $A\subset \Lambda^\vee$ such that for all $x\in \Lambda^\vee=\Hom(\Lambda,\mathbb Z)$ there is some $x'\in A$ such that $x-x'\in N\Lambda^\vee$ and for all $i=1,\ldots,m$, the numbers $x'(\lambda_i)$ and $(x-x')(\lambda_i)$ have the same sign, i.e.~are both nonnegative or both nonpositive.
\end{lemma}

\begin{proof} It suffices to prove the statement for all $x$ such that $\lambda_i(x)\geq 0$ for all $i$; indeed, applying this variant to all $\pm \lambda_i$, one gets the full statement.

Thus, consider the submonoid $\Lambda^\vee_+\subset \Lambda^\vee$ of all $x$ that pair nonnegatively with all $\lambda_i$. This is a finitely generated monoid by standard results; let $y_1,\ldots,y_M$ be a set of generators. Then we can take for $A$ all sums $n_1y_1+\ldots+n_My_M$ where all $n_j\in \{0,\ldots,N-1\}$.
\end{proof}

Now we have the key lemma:

\begin{lemma}\label{lem:key} Let $\Lambda$ be a polyhedral lattice. Then for all positive integers $N$ there is a constant $d$ such that for all $c>0$ one can write any $x\in \Hom(\Lambda,\overline{\mathcal M}_{r'}(S))_{\leq c}$ as
\[
x=x_1+\ldots+x_N
\]
where all $x_i\in \Hom(\Lambda,\overline{\mathcal M}_{r'}(S))_{\leq c/N+d}$.
\end{lemma}

\begin{proof} Pick $\lambda_1,\ldots,\lambda_m\in \Lambda$ generating the norm. We fix a finite subset $A\subset \Lambda^\vee$ satisfying the conclusion of the previous lemma. Write
\[
x=\sum_{n\geq 1, s\in S} x_{n,s} T^n [s]
\]
with $x_{n,s}\in \Lambda^\vee$. Then we can decompose
\[
x_{n,s} = N x_{n,s}^0 + x_{n,s}^1
\]
where $x_{n,s}^1\in A$ and we have the same-sign property of the last lemma. Letting $x^0 = \sum_{n\geq 1, s\in S} x_{n,s}^0 T^n [s]$, we get a decomposition
\[
x = Nx^0 + \sum_{a\in A} a x_a
\]
with $x_a\in \overline{\mathcal M}_{r'}(S)$ (with the property that in the
basis given by the $T^n [s]$, all coefficients are $0$ or $1$). Crucially,
we know that for all $i=1,\ldots,m$, we have
\[
||x(\lambda_i)|| = N ||x^0(\lambda_i)|| + \sum_{a\in A} |a(\lambda_i)| ||x_a(\lambda_i)||
\]
by using the same sign property of the decomposition.

Using this decomposition of $x$, we decompose each term into $N$ summands. This is trivial for the first term $Nx^0$, and each summand of the second term reduces to the similar problem for $\Lambda=\mathbb Z$. In that case, one can take $d=1$, as follows by decomposing any sum with terms of size at most $1$ into $N$ such partial sums whose sums differ by at most $1$. (It follows that in general one can take for $d$ the supremum over all $i$ of $\sum_{a\in A} |a(\lambda_i)|$.)
\end{proof}



\begin{proof}[Proof of Theorem~\ref{thm:explicit3}] We argue by induction on $m$, so assume the result for $m-1$ (this is no assumption for $m=0$, so we do not need an induction start). This gives us some $k>1$ for which the statement of Theorem~\ref{thm:explicit3} holds true for $m-1$; if $m=0$, simply take any $k>1$. In the proof below, we will increase $k$ further in a way that depends only on $m$ and $r$. After this modified choice of $k$, we fix $\epsilon$ and $k_0$ as provided by Proposition~\ref{prop:key}. Moreover, we let $k'$ be the supremum of $k_0$ and the $c_i'$ from Lemma~\ref{lem:basehomotopy} (and~\ref{lem:homotopyNelements}) for $i=0,\ldots,m$. Finally, choose a positive integer $b$ so that $2k'(\tfrac r{r'})^b\leq \epsilon$, and let $N$ be the minimal power of $2$ that satisfies
\[
k'/N\leq (r')^b.
\]
Then in particular $r^bN\leq \frac 2{k'}(\tfrac{r}{r'})^b\leq \epsilon$.

We consider the diagonal embedding
\[
\Lambda\hookrightarrow \Lambda' = \Lambda^N,
\]
where we endow $\Lambda'$ with the norm
\[
||(\lambda_1,\ldots,\lambda_N)||_{\Lambda'} = \tfrac 1N(||\lambda_1||_\Lambda+\ldots+||\lambda_N||_\Lambda).
\]
For any $m\geq 1$, let $\Lambda'^{(m)}$ be given by $\Lambda'^m / \Lambda\otimes (\mathbb Z^m)_{\sum=0}$; then $\Lambda'^{(\bullet)}$ is cosimplicial polyhedral lattice, the \v{C}ech conerve of $\Lambda\to \Lambda'$. For $m=0$, we set $\Lambda'^{(0)} = \Lambda$. It is clear that all of these are polyhedral lattices.

In particular, for any $c>0$, we have
\[
\Hom(\Lambda'^{(m)},\overline{\mathcal M}_{r'}(S))_{\leq c} = \Hom(\Lambda',\overline{\mathcal M}_{r'}(S))_{\leq c}^{m/\Hom(\Lambda,\overline{\mathcal M}_{r'}(S))_{\leq c}},
\]
the $m$-fold fibre product of $\Hom(\Lambda',\overline{\mathcal M}_{r'}(S))_{\leq c}$ over $\Hom(\Lambda,\overline{\mathcal M}_{r'}(S))_{\leq c}$; and
\[
\Hom(\Lambda',\overline{\mathcal M}_{r'}(S))_{\leq c} = \Hom(\Lambda,\overline{\mathcal M}_{r'}(S))_{\leq c/N}^N,
\]
with the map to $\Hom(\Lambda,\overline{\mathcal M}_{r'}(S))_{\leq c}$ given by the sum map.

Consider the collection of double complexes $C_{\Lambda'^{(\bullet)},c}^\bullet$ associated to this cosimplicial polyhedral lattice by Dold-Kan. Up to rescaling the norms in the complex for $\Lambda'^{(m)}$ by a universal constant (something like $(m+2)!$), the differentials are strictly compatible with norms (as they are an alternating sum of $m+1$ face maps, all of which are of norm $\leq 1$), so this collection of normed double complexes is admissible. By induction, the first condition of Proposition~\ref{prop:key} is satisfied for all $c\geq c_0$ with $c_0$ large enough (depending on $\Lambda$ but not $V$ or $S$). By Lemma~\ref{lem:key}, and noting that $\Hom(\Lambda'^{(\bullet)},\overline{\mathcal M}_{r'}(S))_{\leq c}$ is the \v{C}ech nerve of
\[
\Hom(\Lambda,\overline{\mathcal M}_{r'}(S))_{\leq c/N}^N\xrightarrow{\sum} \Hom(\Lambda,\overline{\mathcal M}_{r'}(S))_{\leq c},
\]
also the second condition is satisfied, with $k$ the maximum of the previous $k$ and some constant depending only on $m$ and $r$, provided we take $c_0$ large enough so that $(k-1)r'c_ic_0/N$ is at least the $d$ of Lemma~\ref{lem:key} for all $i=0,\ldots,m$ (so this choice of $c_0$ again depends on $\Lambda$). Indeed, then one can splice a surjection of profinite sets between the maps
\[
\Hom(\Lambda,\overline{\mathcal M}_{r'}(S))_{\leq c_ic/N}^{Na}\to\Hom(\Lambda,\overline{\mathcal M}_{r'}(S))_{\leq c_ic}^a
\]
and
\[
\Hom(\Lambda,\overline{\mathcal M}_{r'}(S))_{\leq kc_ic/N}^{Na}\to \Hom(\Lambda,\overline{\mathcal M}_{r'}(S))_{\leq kc_ic}^a,
\]
and so the transition map between the columns of that double complex factors over a similar complex arising from a simplicial hypercover of profinite sets, so the constants are bounded by Proposition~\ref{prop:normedcompletion}, Lemma~\ref{lem:Tinv}, and Proposition~\ref{prop:snakelemma}. At this point, we have finalized our choice of $k$ (and, as promised, this choice depended only on $m$ and $r$), and so we also finalized the constants $\epsilon$, $k'$ and $N$ from the first paragraph of the proof.

Finally, to check the third condition, we use Lemma~\ref{lem:homotopyNelements} to find, in degrees $\leq m$, a homotopy between the two maps from the first row
\[
\widehat{V}(\Hom(\Lambda,\overline{\mathcal M}_{r'}(S))_{\leq c})^{T^{-1}}\to \widehat{V}(\Hom(\Lambda,\overline{\mathcal M}_{r'}(S))_{\leq c_1c}^2)^{T^{-1}}\to \ldots
\]
to the second row
\[
\widehat{V}(\Hom(\Lambda,\overline{\mathcal M}_{r'}(S))_{\leq c/N}^N)^{T^{-1}}\to \widehat{V}(\Hom(\Lambda,\overline{\mathcal M}_{r'}(S))_{\leq c_1c/N}^{2N})^{T^{-1}}\to \ldots
\]
respectively induced by the addition $\Hom(\Lambda,\overline{\mathcal M}_{r'}(S))_{\leq c/N}^N\to \Hom(\Lambda,\overline{\mathcal M}_{r'}(S))_{\leq c}$ (which is the map that forms part of the double complex), and the map that is the sum of the $N$ maps induced by the $N$ projection maps
\[
\Hom(\Lambda,\overline{\mathcal M}_{r'}(S))_{\leq c/N}^N\to \Hom(\Lambda,\overline{\mathcal M}_{r'}(S))_{\leq c/N}\subset \Hom(\Lambda,\overline{\mathcal M}_{r'}(S))_{\leq c}.
\]
By Lemma~\ref{lem:homotopyNelements}, we can find this homotopy between the complex for $k'c$ and the complex for $c$, by our choice of $k'\geq c_i'$ for $i=0,\ldots,m$. As $N$ is fixed, the homotopy is the universal homotopy from Lemma~\ref{lem:homotopyNelements}, and in particular its norm is bounded by some universal constant $H$.

Finally, it remains to establish the estimate \eqref{eq:homotopicmapsmall} on the homotopic map. We note that this takes $x\in \widehat{V}(\Hom(\Lambda,\overline{\mathcal M}_{r'}(S))_{\leq k'^2c_ic}^{a_i})^{T^{-1}}$ (with $i=q$ in the notation of \eqref{eq:homotopicmapsmall}) to the element
\[
y\in \widehat{V}(\Hom(\Lambda,\overline{\mathcal M}_{r'}(S))_{\leq k'c_ic/N}^{Na_i})^{T^{-1}}
\]
that is the sum of the $N$ pullbacks along the $N$ projection maps $\Hom(\Lambda,\overline{\mathcal M}_{r'}(S))_{\leq k'c_ic/N}^{Na_i}\to \Hom(\Lambda,\overline{\mathcal M}_{r'}(S))_{\leq k'^2c_ic}^{a_i}$. We note that these actually take image in $\Hom(\Lambda,\overline{\mathcal M}_{r'}(S))_{\leq c_ic}^{a_i}$ as $N\geq k'$, so this actually gives a well-defined map
\[
\widehat{V}(\Hom(\Lambda,\overline{\mathcal M}_{r'}(S))_{\leq c_ic}^{a_i})^{T^{-1}}\to \widehat{V}(\Hom(\Lambda,\overline{\mathcal M}_{r'}(S))_{\leq k'c_ic/N}^{Na_i})^{T^{-1}}.
\]
We need to see that this map is of norm $\leq \epsilon$. Now note that by our choice of $N$, we actually have $k'c_ic/N\leq (r')^b c_ic$, so this can be written as the composite of the restriction map
\[
\widehat{V}(\Hom(\Lambda,\overline{\mathcal M}_{r'}(S))_{\leq c_ic}^{a_i})^{T^{-1}}\to \widehat{V}(\Hom(\Lambda,\overline{\mathcal M}_{r'}(S))_{\leq (r')^b c_ic}^{a_i})^{T^{-1}}
\]
and
\[
\widehat{V}(\Hom(\Lambda,\overline{\mathcal M}_{r'}(S))_{\leq (r')^b c_ic}^{a_i})^{T^{-1}}\to \widehat{V}(\Hom(\Lambda,\overline{\mathcal M}_{r'}(S))_{\leq k'c_ic/N}^{Na_i})^{T^{-1}}.
\]
The first map has norm exactly $r^b$, by $T^{-1}$-invariance, and as multiplication by $T$ scales the norm with a factor of $r$ on $\widehat{V}$.\footnote{Here is where we use $r'>r$, ensuring different scaling behaviour of the norm on source and target.} The second map has norm at most $N$ (as it is a sum of $N$ maps of norm $\leq 1$). Thus, the total map has norm $\leq r^bN$. But by our choice of $N$, we have $r^bN\leq \epsilon$, giving the result.

Thus, we can apply Proposition~\ref{prop:key}, and get the desired $\leq \max(k'^2,2k_0H)$-exactness in degrees $\leq m$ for $c\geq c_0$, where $k'$, $k_0$ and $H$ were defined only in terms of $k$, $m$, $r'$ and $r$, while $c_0$ depends on $\Lambda$ (but not on $V$ or $S$). This proves the inductive step.
\end{proof}

\begin{question} Can one make the constants explicit, and how large are they?\footnote{A back of the envelope calculation seems to suggest that $k$ is roughly doubly exponential in $m$, and that $N$ has to be taken of roughly the same magnitude.} Modulo the Breen-Deligne resolution, all the arguments give in principle explicit constants; and actually the proof of the existence of the Breen-Deligne resolution should be explicit enough to ensure the existence of bounds on the $c_i$ and $c_i'$.
\end{question}

This completes the proof of all results announced so far.

\newpage

\section*{Appendix to Lecture IX: Some normed homological algebra}

In this appendix, we gather a few results about homological algebra with normed abelian groups, the proofs of which are just obtained by keeping track of constants in the standard proofs.


\begin{proposition}\label{prop:snakelemma} Let $M^\bullet_c$ and $M'^\bullet_c$ be two admissible collections of complexes of complete normed abelian groups, where $c\geq c_0$. Let $f^\bullet_c: M^\bullet_c\to M'^\bullet_c$ be a collection of maps between these collections of complexes that is strictly compatible with the norm and commutes with restriction maps, and assume that it satisfies
\[
||\mathrm{res}^i_{kc,c}(x)||_{M^i_c}\leq k||f^i_{kc}(x)||_{M'^i_{kc}}
\]
for all $i=0,\ldots,m+1$ and all $x\in M^i_{kc}$. Let $N^\bullet_c=M'^\bullet_c/\overline{M^\bullet_c}$ (which equals the completion of $M'^{\bullet}_c/M^\bullet_c$) be the collection of quotient complexes, with the quotient norm; this is again an admissible collection of complexes.

Assume that $M^\bullet_c$ and $M'^\bullet_c$ are $\leq k$-exact in degrees $\leq m$ for $c\geq c_0$. Then $N^\bullet_c$ is $\leq \max(k^4,k^3+k+1)$-exact in degrees $\leq m-1$ for $c\geq c_0$.
\end{proposition}

\begin{proof} We make the following preliminary observation. Take any $i=0,\ldots,m+1$ and $m'_{kc}\in M'^i_{kc}$ with image $n_{kc}\in N^i_{kc}$. By the definition of the quotient norm, for any $\epsilon>0$ we can find some $m_{kc}\in M^i_{kc}$ such that $||m'_{kc}-f^i_{kc}(m_{kc})||\leq ||n_{kc}||+\epsilon$. We would like to replace this by the stronger assertion that we can find $m_{kc}\in M^i_{kc}$ such that
\[
||m'_{kc}-f^i_{kc}(m_{kc})||\leq (1+\epsilon)||n_{kc}||.
\]
This is obviously possible as long as $||n_{kc}||>0$, but in case $||n_{kc}||=0$, it may not be possible, because $M^\bullet_c\to M'^\bullet_c$ may not have closed image.

However, we claim that, letting $m'_c\in M'^i_c$ be the restriction of $m'_{kc}\in M'^i_{kc}$, with image $n_c\in N^i_c$, we can always find some $m_c\in M^i_c$ such that
\[
||m'_c-f^i_c(m_c)||\leq (1+\epsilon)||n_{kc}||.
\]
By the above, we only need to prove this when $||n_{kc}||=0$. Choose a sequence $m_{kc,0},m_{kc,1},\ldots$ in $M^i_{kc}$ such that $||m'_{kc}-f^i_{kc}(m_{kc,j})||\to 0$ for $j\to \infty$. In particular, $||f^i_{kc}(m_{kc,j}-m_{kc,j'})||\to 0$ for $j,j'\to \infty$. By the displayed bound in the statement of the proposition, this ensure that $||m_{c,j}-m_{c,j'}||\to 0$ where $m_{c,j}\in M^i_c$ is the image of $m_{kc,j}$. Thus, we get a Cauchy sequence in $M^i_c$ whose limit $m_c\in M^i_c$ will satisfy $||m'_c-f^i_c(m_c)||=0$ (i.e.~$m'_c=f^i_c(m_c)$).

Now we start the proof of the proposition. Let $n^i_{k^4c}\in N^i_{k^4c}$ for $i\leq m-1$, with image $n^{i+1}_{k^4c}\in N^{i+1}_{k^4c}$, and let $C:=||n^{i+1}_{k^4c}||_{N^{i+1}_{k^4c}}$. We need to find an element $n^i_c\in N^{i-1}_c$ such that
\[
||n^i_c - d^{i-1}_{N,c}(n^{i-1}_c)||_{N^i_c}\leq (k^3+k+1)C,
\]
where we change the subscript when applying restriction maps.

Pick any preimage $m'^i_{k^4c}\in M'^i_{k^4c}$ of $n^i_{k^4c}$, and let $m'^{i+1}_{k^4c}\in M'^{i+1}_{k^4c}$ be its image. By the preliminary observation, we can find $m^{i+1}_{k^3c}\in M^{i+1}_{k^3c}$ such that
\[
m'^{i+1}_{k^3c} = f^{i+1}_{k^3c}(m^{i+1}_{k^3c}) + m''^{i+1}_{k^3c}
\]
with $||m''^{i+1}_{k^3c}||_{M'^{i+1}_{k^3c}}\leq (1+\epsilon)C$, where we choose $\epsilon$ so that $(k^3+k)(1+\epsilon)\leq k^3+k+1$.

Let $m^{i+2}_{k^3c}\in M^{i+2}_{k^3c}$ be the image of $m^{i+1}_{k^3c}$. Applying the differential to the last displayed equation, and using that this kills $m'^{i+1}_{k^3c}$, and that $f^\bullet_{k^3c}$ is a map of complexes, we see that
\[
f^{i+2}_{k^3c}(m^{i+2}_{k^3c}) = -m''^{i+2}_{k^3c},
\]
where similarly $m''^{i+2}_{k^3c}$ is the differential of $m''^{i+1}_{k^3c}$. We get
\[\begin{aligned}
||m^{i+2}_{k^2c}||_{M^{i+2}_{k^2c}}&\leq k||f^{i+2}_{k^3c}(m^{i+2}_{k^3c})||_{M'^{i+2}_{k^3c}} = k||m''^{i+2}_{k^3c}||_{M'^{i+2}_{k^3c}}\\
&\leq k||m'^{i+1}_{k^3c}||_{M'^{i+1}_{k^3c}}\leq k(1+\epsilon)C.
\end{aligned}\]
On the other hand, we can find some $m^i_{kc}\in M^i_{kc}$ such that
\[
||m^{i+1}_{kc}-d^i_{kc}(m^i_{kc})||\leq k||m^{i+2}_{k^2c}||_{M^{i+2}_{k^2c}}\leq k^2(1+\epsilon)C.
\]
Now let $m'^i_{kc,\mathrm{new}} = m'^i_{kc}-f^i_{kc}(m^i_{kc})\in M'^i_{kc}$; this is a lift of $n^i_{kc}$. Then the image $m'^{i+1}_{kc,\mathrm{new}}$ in $M'^{i+1}_{kc}$ satisfies
\[
m'^{i+1}_{kc,\mathrm{new}} = m'^{i+1}_{kc}-f^{i+1}_{kc}(m^{i+1}_{kc}) + f^{i+1}_{kc}(m^{i+1}_{kc}-d^i_{kc}(m^i_{kc})) = m''^{i+1}_{kc} + f^{i+1}_{kc}(m^{i+1}_{kc}-d^i_{kc}(m^i_{kc})).
\]
In particular,
\[
||m'^{i+1}_{kc,\mathrm{new}}||_{M'^{i+1}_{kc}}\leq (1+\epsilon)C+ k^2(1+\epsilon)C.
\]
Now we can find $m'^{i-1}_c\in M'^{i-1}_c$ such that
\[
||m'^i_{c,\mathrm{new}} - d'^{i-1}_c(m'^{i-1}_c)||_{M'^i_c}\leq k||m'^{i+1}_{kc,\mathrm{new}}||_{M'^{i+1}_{kc}}\leq (k^3+k)(1+\epsilon)C.
\]
In particular, letting $n^{i-1}_c\in N^{i-1}_c$ be the image of $m'^{i-1}_c$, we get
\[
||n^i_c - d^{i-1}_{N,c}(n^{i-1}_c)||_{N^i_c}\leq (k^3+k)(1+\epsilon)C,
\]
so by our choice of $\epsilon$ this gives the desired result.
\end{proof}

We need the following results about the Breen-Deligne resolution for normed abelian groups. Let us consider here abelian groups $M$ (in any topos) equipped with an increasing filtration $M_{\leq c}\subset M$ by subobjects indexed by the positive real numbers, such that $0\in M_{\leq c}$, $-M_{\leq c} = M_{\leq c}$ and $M_{\leq c}+M_{\leq c'}\subset M_{\leq c+c'}$; we need no further conditions. Let us call these pseudo-normed abelian groups.

Fix a choice of a functorial Breen-Deligne resolution
\[
C(M):\ldots \to \mathbb Z[M^{a_i}]\to\ldots\to\mathbb
Z[M^{a_1}]\to\mathbb Z[M^{a_0}]\to M\to 0
\]
of an abelian group $M$; purely for notational convenience, we can and do assume that each term is of the form $\mathbb Z[M^{a_i}]$ (as opposed to a finite direct sum of such). The possibility of doing this follows from the proof of \cite[Theorem 4.10]{Condensed}, noting that a functor of the form $A\mapsto \mathbb Z[A^n]\oplus \mathbb Z[A^m]$ admits a surjection from the functor $A\mapsto \mathbb Z[A^{n+m}]\oplus \mathbb Z$; this gives a resolution where all terms are of the form $\mathbb Z[A^{a_i}]\oplus \mathbb Z^m$. Now pass to the quotient of these complexes corresponding to the map $0\to A$; this gives a complex all of whose terms are of the form $\mathbb Z[A^{a_i}]/\mathbb Z$. Noting that $\mathbb Z[A^{a_i}]$ is functorially isomorphic to $\mathbb Z[A^{a_i}]/\mathbb Z\oplus \mathbb Z$ (via splitting $0\to A^{a_i}\to 0$), we can then add an acyclic complex of $\mathbb Z$'s in each degree to get a resolution all of whose terms are of the form $\mathbb Z[A^{a_i}]$.

\begin{lemma}\label{lem:constantsdeligne} There are universal constants $c_0=1,c_1,c_2,\ldots$ so that the Breen-Deligne resolution admits the subcomplex
\[
C(M)_{\leq c}: \ldots \to \mathbb Z[M^{a_i}_{\leq c_ic}]\to\ldots\to\mathbb
Z[M^{a_1}_{\leq c_1c}]\to\mathbb Z[M^{a_0}_{\leq c}]
\]
for all pseudo-normed abelian group objects in any topos as above, and all $c>0$.
\end{lemma}

\begin{proof} Each differential in the Breen-Deligne resolution is a finite sum of maps induced by maps $M^{a_{i+1}}\to M^{a_i}$ given by some $a_i\times a_{i+1}$-matrix of integers. Given $c_i$, one can thus find some $c_{i+1}$ so that $M^{a_{i+1}}_{\leq c_{i+1}c}$ maps into $M^{a_i}_{\leq c_ic}$ for each of those finitely many maps, which gives the claim.
\end{proof}

We also need some homotopies. More precisely, we start with the following homotopy.

\begin{lemma}\label{lem:basehomotopy} For an abelian group $M$, the maps $\sigma_1,\sigma_2$ from
\[
C(M^2): \ldots \to \mathbb Z[M^{2a_i}]\to\ldots\to\mathbb Z[M^{2a_1}]\to\mathbb Z[M^{2a_0}]
\]
to
\[
C(M): \ldots \to \mathbb Z[M^{a_i}]\to\ldots\to\mathbb Z[M^{a_1}]\to\mathbb Z[M^{a_0}],
\]
induced by addition $M^2\to M$, respectively the sum of the two maps induced by two projections $M^2\to M$, are homotopic, via some functorial homotopy
\[
h_i: \mathbb Z[M^{2a_i}]\to \mathbb Z[M^{a_{i+1}}].
\]

If $M$ is a pseudo-normed abelian group object in any topos, then $\sigma_1$ and $\sigma_2$ are well-defined as maps of complexes from
\[
C(M^2)_{\leq c/2}: \ldots \to \mathbb Z[M^{2a_i}_{\leq c_ic/2}]\to\ldots\to\mathbb
Z[M^{2a_1}_{\leq c_1c/2}]\to\mathbb Z[M^{2a_0}_{\leq c/2}]
\]
to
\[
C(M)_{\leq c}: \ldots \to \mathbb Z[M^{a_i}_{\leq c_ic}]\to\ldots\to\mathbb
Z[M^{a_1}_{\leq c_1c}]\to\mathbb Z[M^{a_0}_{\leq c}]
\]
for all $c>0$. In that case, for all $i\geq 0$ there are universal constants $c_i'$ such that $h_i$ defines well-defined maps
\[
\mathbb Z[M^{2a_i}_{\leq c_ic/2}]\to \mathbb Z[M^{a_{i+1}}_{\leq
c_i'c_{i+1}c}]
\]
for all $c>0$.
\end{lemma}

\begin{proof} This is a consequence of the proof of the existence of the Breen-Deligne resolution, proved in the same way as \cite[Proposition 4.17]{Condensed}. The existence of the constants $c_i'$ is again formal, as in the last lemma.
\end{proof}

Now we need the following generalization to adding $N$ elements.

\begin{lemma}\label{lem:homotopyNelements} Let $N$ be a power of $2$. The maps of complexes $\sigma_1,\sigma_2$ from
\[
C(M^N): \ldots \to \mathbb Z[M^{Na_i}]\to\ldots\to\mathbb Z[M^{Na_1}]\to\mathbb Z[M^{Na_0}]
\]
to
\[
C(M): \ldots \to \mathbb Z[M^{a_i}]\to\ldots\to\mathbb Z[M^{a_1}]\to\mathbb Z[M^{a_0}],
\]
induced by addition $M^N\to M$, respectively the sum of the $N$ maps induced by the $N$ projections $M^N\to M$, are homotopic, via some functorial homotopy
\[
h_i^N: \mathbb Z[M^{Na_i}]\to \mathbb Z[M^{a_{i+1}}]
\]
which moreover satisfies the following bound, with the same constants $c_0',c_1',\ldots$ as in the previous lemma:

If $M$ is a pseudo-normed abelian group object in any topos, then $\sigma_1$ and $\sigma_2$ are well-defined as maps of complexes from
\[
C(M^N)_{\leq c/N}: \ldots \to \mathbb Z[M^{Na_i}_{\leq c_ic/N}]\to\ldots\to\mathbb Z[M^{Na_1}_{\leq c_1c/N}]\to\mathbb Z[M^{Na_0}_{\leq c/N}]
\]
to
\[
C(M)_{\leq c}: \ldots \to \mathbb Z[M^{a_i}_{\leq c_ic}]\to\ldots\to\mathbb Z[M^{a_1}_{\leq c_1c}]\to\mathbb Z[M^{a_0}_{\leq c}]
\]
for all $c>0$. In that case, $h_i^N$ defines well-defined maps
\[
\mathbb Z[M^{Na_i}_{\leq c_ic/N}]\to \mathbb Z[M^{a_{i+1}}_{\leq c_i'c_{i+1}c}]
\]
for all $c>0$.
\end{lemma}

\begin{proof} Let $N=2^m$. For each $j=0,\ldots,m-1$, the two maps from $C(M^{2^{j+1}})$ to $C(M^{2^j})$ from the previous lemma are homotopic, and we use the homotopy from that lemma. Composing homotopies (which amounts concretely to a certain sum) we get the desired homotopy from $C(M^{2^m})$ to $C(M)$. It follows directly from this construction that the constants $c_i'$ are unchanged.
\end{proof}

\newpage

\bibliographystyle{amsalpha}

\bibliography{Analytic}

\end{document}
