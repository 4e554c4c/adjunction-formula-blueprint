\section{Key technical result}

Now we state the following result, which is the key technical result on our to the main goal.

\begin{theorem}
  \label{first_target}
  \lean{thm94}
  \leanok
  \uses{BD_system}
  Let $\BD = (n,f,h)$ be a Breen--Deligne package,
  and let $\kappa = (\kappa_0, \kappa_1, \kappa_2, \dots)$ be a sequence of constants in $\mathbb R_{\ge 0}$
  that is $\BD$-suitable.
	Fix radii $1>r'>r>0$.
  For any $m$ there is some $k$ and $c_0$ such that for all profinite sets $S$ and all $r$-normed $\mathbb Z[T^{\pm 1}]$-modules $V$,
  the system of complexes
  \[
    C^{\BD}_{\kappa}(\overline{\mathcal M}_{r'}(S))_c^\bullet \colon
    \widehat{V}(\overline{\mathcal M}_{r'}(S)_{\leq c})^{T^{-1}} \to
    \widehat{V}(\overline{\mathcal M}_{r'}(S)_{\leq \kappa_1c}^2)^{T^{-1}}
    \to \ldots
  \]
  is $\leq k$-exact in degrees $\leq m$ for $c\geq c_0$.
\end{theorem}

We will prove Theorem~\ref{first_target} by induction on $m$.
Unfortunately, the induction requires us to prove a stronger statement.

\begin{theorem}
  \label{explicit}
  \uses{Hom, BD_system, Mbar_with_Tinv}
  \lean{thm95''.profinite}
  \leanok
  Fix radii $1>r'>r>0$. For any $m$ there is some $k$
  such that for all polyhedral lattices $\Lambda$
  there is a constant $c_0(\Lambda)>0$
  such that for all profinite sets $S$
  and all $r$-normed $\mathbb Z[T^{\pm 1}]$-modules $V$,
  the system of complexes
  \[
  C_{\Lambda,c}^\bullet \colon
  \widehat{V}(\Hom(\Lambda,\overline{\mathcal M}_{r'}(S))_{\leq c})^{T^{-1}} \to
  \widehat{V}(\Hom(\Lambda,\overline{\mathcal M}_{r'}(S))_{\leq \kappa_1c}^2)^{T^{-1}} \to \ldots
  \]
  is $\leq k$-exact in degrees $\leq m$ for $c\geq c_0(\Lambda)$.
\end{theorem}

\begin{proof}[{Proof of Theorem~\ref{first_target}}]
  \proves{first_target}
  \uses{explicit}
  \leanok
  Use $\Lambda = \mathbb Z$, and the isomorphism $\Hom(\mathbb Z, A) \cong A$.
\end{proof}

\textbf{A word on universal constants}:
We fix once and for all, the constants $0 < r < r' \le 1$
a Breen--Deligne package $\BD$,
and a sequence of positive constants $\kappa$ that is very suitable for $(\BD, r, r')$.
Once the full proof is formalized,
we can come back to this place and write a bit more about the other constants.

\textbf{The global strategy}
of the proof is to construct a system of double complexes
such that its first row is the system $C_{\Lambda, \bullet}^\bullet$
occuring in Theorem~\ref{explicit}.
We can then verify the conditions to Proposition~\ref{spectral}
and conclude from there.
For the time being, we will let $M$ denote
an arbitrary profinitely filtered pseudo-normed group with action of $T^{-1}$,
and whenever needed we can specialize to $M = \overline{\mathcal M}_{r'}(S)$.

\textbf{Further choices of constants}:
We will argue by induction on $m$, so assume the result for $m-1$
(this is no assumption for $m=0$, so we do not need an induction start).
This gives us some $k>1$ for which the statement of Theorem~\ref{explicit} holds true for $m-1$;
if $m=0$, simply take any $k>1$.
In the proof below, we will increase $k$ further in a way that depends only on $m$ and $r$.
After this modified choice of $k$, we fix $\epsilon$ and $k_0$ as provided by Proposition~\ref{spectral}.
Fix a sequence $(\kappa'_i)_i$ of nonnegative reals that is adept to $(\BD, \kappa)$.
(Such a sequence exists by Lemma~\ref{exists_adept}.)
Moreover, we let $k'$ be the supremum of $k_0$ and the $c_i'$ for $i=0,\ldots,m+1$.
Finally, choose a positive integer $b$ so that $2k'(\tfrac r{r'})^b\leq \epsilon$,
and let $N$ be the minimal power of $2$ that satisfies
\[
  k'/N\leq (r')^b.
\]
Then in particular $r^bN\leq 2k'(\tfrac{r}{r'})^b\leq \epsilon$.

\begin{definition}
  \label{double_complex}
  \lean{thm95.double_complex}
  \uses{Hom, cosimplicial-lattice, BD_system}
  \leanok
  Let $\Lambda^{(\bullet)}$ be the cosimplicial polyhedral lattice of
  Definition~\ref{cosimplicial-lattice},
  and recall from \ref{Hom} that $\Hom(\Lambda^{(m)}, M)$ is a
  profinitely filtered pseudo-normed group with action of $T^{-1}$.

  Hence $\Hom(\Lambda^{(\bullet)}, M)$ is a simplicial
  profinitely filtered pseudo-normed group with action of $T^{-1}$.

  Now apply the construction of the system of complexes from
  Definition~\ref{BD_system} to obtain a cosimplicial system of complexes
  \[
    C^{\BD}_{\kappa}(\Hom(\Lambda^{(\bullet)}, M))_\bullet^\bullet.
  \]
  Now take the alternating face map cochain complex
  to obtain a system of double complexes, whose objects are
  \[
    \widehat{V}(\Hom(\Lambda^{(m)},M)_{\leq \kappa_ic}^{n_i})^{T^{-1}}.
  \]
  As final step, rescale the norm on the object in row $m$ by $m!$,
  so that all columns become admissible:
  the vertical differential from row $m$ to row $m+1$
  is an alternating sum of $m+1$ maps that are all norm-nonincreasing.
\end{definition}

\begin{lemma}
  \label{canonical_iso_1}
  \uses{double_complex}
  \lean{PolyhedralLattice.Hom_cosimplicial_zero_iso}
  \leanok
  In particular, for any $c>0$, we have
  \[
    \Hom(\Lambda',M)_{\leq c} = \Hom(\Lambda,M)_{\leq c/N}^N,
  \]
  with the map to $\Hom(\Lambda,M)_{\leq c}$ given by the sum map.
\end{lemma}

\begin{proof}
  \leanok
  Omitted (but done in Lean).
\end{proof}

\begin{lemma}
  \label{canonical_iso_2}
  \uses{double_complex}
  \lean{thm95.Cech_nerve_level_iso}\leanok
  Similarly, for any $c>0$, we have
  \[
    \Hom(\Lambda'^{(m)},M)_{\leq c} =
    \Hom(\Lambda',M)_{\leq c}^{m/\Hom(\Lambda,M)_{\leq c}},
  \]
  the $m$-fold fibre product of $\Hom(\Lambda',M)_{\leq c}$
  over $\Hom(\Lambda,M)_{\leq c}$.
\end{lemma}

\begin{proof}
  \leanok
  Omitted (but done in Lean).
\end{proof}

\begin{lemma}
  \label{row_one_iso}
  \uses{canonical_iso_1}
  \lean{thm95.mul_rescale_iso_row_one}
  \leanok
  There is a canonical isomorphism between the first row of the double complex
  \[
    C^{\BD}_{\kappa}(\Hom(\Lambda^{(1)}, M))^\bullet
  \]
  and
  \[
    C^{N \otimes \BD}_{\kappa/N}(\Hom(\Lambda, M))^\bullet
  \]
  which identifies the map induced by
  the diagonal embedding $\Lambda \to \Lambda' = \Lambda^{(1)}$
  with the map induced by $\sigma^N \colon N \otimes \BD \to \BD$.
\end{lemma}

\begin{proof}
  \leanok
  Omitted (but done in Lean).
\end{proof}

\begin{proposition}
  \label{cechcover-exact}
  \uses{CLC}
  \lean{prop819}
  \leanok
  Let $\pi : X \to B$ be a surjective morphism of profinite sets,
  and let $S_\bullet \to S_{-1}$, $S_{-1} := B$, be its augmented \v{C}ech nerve.
  Let $V$ be a semi-normed group.
  Then the complex
  \[
    0\to \widehat{V}(S_{-1})\to \widehat{V}(S_0)\to \widehat{V}(S_1)\to \cdots
  \]
  is exact.
  Furthermore, for all $\epsilon > 0$ and $f \in \ker(\widehat{V}(S_{m}) \to \widehat{V}(S_{m+1}))$,
  there exists some $g\in \widehat{V}(S_{m-1})$ such that $d(g) = f$ and $‖g‖\leq (1+\epsilon) \cdot ‖f‖$.
  In other words, the complex is normed exact in the sense of Definition~\ref{norm_exact_complex}.
\end{proposition}

\begin{proof}
  \uses{lem:controlled_exactness}\leanok
  We argue similarly to~\cite[Theorem 3.3]{Condensed}, as follows.
  By applying Lemma~\ref{lem:controlled_exactness}, we first reduce to a statement which does not involve $\epsilon$ or completions.
  Explicitly, we must show that
  \[ 0 \to V(S_{1}) \to V(S_{0}) \to V(S_{1}) \to \cdots \]
  is exact, and that whenever $f \in \ker(V(S_{m}) \to V(S_{m+1}))$, there exists $g \in V(S_{m-1})$ such that $‖g‖\leq ‖f‖$ and $d(g) = f$.
  The map $V(S_{-1}) \to V(S_{0})$ is the one induced by $S_0 \to S_{-1}$ which agrees with $X \to B$.
  Since $X \to B$ is surjective, we easily see that $V(S_{-1}) \to V(S_{0})$ is injective.

  If $X$ and $B$ are finite, then the remaining assertions follow from the existence of a splitting $\sigma : B \to X$ of $\pi : X \to B$, as follows.
  The map $\sigma$ provides maps $S_{m} \to S_{m+1}$ for all $m \geq -1$, defined explicitly as
  \[ (a_{0},\ldots,a_{m}) \mapsto (\sigma (\pi (a_{0})), a_{0},\ldots,a_{m}) \]
  if $m \geq 0$ and simply as $\sigma$ if $m = -1$.
  Here, for $m \geq 0$, we have identified $S_{m}$ with the $m+1$-fold fibered product $X \times_{B} \cdots \times_{B} X$.
  Applying $V(-)$, these maps induce $h_{m} : V(S_{m+1}) \to V(S_{m})$, which form a contracting homotopy for the complex in question, and which are norm nonincreasing by the definition of $V(-)$.
  If $f \in \ker(V_{m} \to V_{m+1})$ is as above, then $g := h_{m}(f)$ satisfies $d(g) = f$ and $‖g‖\leq ‖f‖$, as required.

  In the general case, write $X = \varprojlim_{i} X_{i}$ where $X_{i}$ vary over the discrete (hence finite) quotients of $X$.
  Since $X \to B$ is surjective, for each $i$ there exists a unique maximal discrete quotient $B_{i}$ of $B$ such that $X \to B$ descends to $X_{i} \to B_{i}$.
  The maps $X_{i} \to B_{i}$ are again surjective, and one has
  \[ (X \to B) = \varprojlim_{i} (X_{i} \to B_{i}). \]
  Let $S_{i,\bullet} \to S_{i,-1}$, $S_{i,-1} := B_{i}$, denote the augmented \v{C}ech nerve of $X_{i} \to B_{i}$.

  The terms in the \v{C}ech nerve are themselves limits, hence we have $S_{m} = \varprojlim_{i} S_{i,m}$, with each $S_{i,m}$ finite.
  The functor $V(-)$, when considered as taking values in abelian groups, sends cofiltered limits to filtered colimits.
  Also, if $f \in V(S_{m})$ is the pullback of $f_{i} \in V(S_{i,m})$, then for a sufficiently small index $j \leq i$, the image of $f : S_{m} \to V$ agrees with the image of $f_{j} : S_{j,m} \to V$, where $f_{j}$ is the image of $f_{i}$ under the map $V(S_{i,m}) \to V(S_{j,m})$ induced by the transition map $S_{j,m} \to S_{i,m}$.

  Now suppose that $f \in \ker(V(S_{m}) \to V(S_{m+1}))$ is given.
  By the discussion above, there exists some $i$ and some $f_{i} \in V(S_{i,m})$ such that $f$ is the pullback of $f_{i}$ with respect to the morphism $S_{m} \to S_{i,m}$ and such that the following additional conditions hold:
  \begin{enumerate}
    \item One has $‖f_{i}‖ = ‖f‖$.
    \item One has $f_{i} \in \ker (V(S_{i,m}) \to V(S_{i,m+1}))$.
  \end{enumerate}
  Let $h_{m} : V(S_{i,m}) \to V(S_{i,m-1})$ be the map constructed in the argument for the finite case $X_{i} \to B_{i}$.
  Put $g_{i} := h_{m} (f_{i})$ and $g$ the image of $g_{i}$ in $V(S_{m-1})$.
  Since the maps $V(S_{i,\bullet}) \to V(S_{\bullet})$ commute with the differentials, we have $d(g) = f$.
  Finally, the map $V(S_{i,m-1}) \to V(S_{m-1})$ is norm nonincreasing as it is induced from $S_{m-1} \to S_{i,m-1}$, so that
  \[ ‖g‖\leq ‖g_{i}‖ \le ‖f_{i}‖ = ‖f‖, \]
  as contended.
\end{proof}

\begin{lemma}
  \label{pre_col_exact}
  \uses{splittable}
  \lean{thm95.FLC_complex.weak_bounded_exact}
  \leanok
  Let $M$ be a profinitely filtered pseudo-normed group with $T^{-1}$-action
  that is $N$-splittable with error term~$d \ge 0$.
  Let $k \ge 1$ be a real number,
  and let $c_0 > 0$ satisfy $d \le \frac{(k - 1) c_0}{N}$.
  For every $c$, consider the Cech nerve of the summation map $M^N_{c/N} \to M_c$.
  By applying the functor $\hat V(\_)$ and taking the alternating face map complex,
  we obtain a system of complexes
  \[
    \hat V(M_{\le c}) \to \hat V(M^N_{\le c/N}) \to \dots
  \]
  This system of complexes
  is weakly $\le k$-exact in degrees $\le m$ and for $c \ge c_0$ with bound~$1$.
\end{lemma}

\begin{proof}
  \uses{cechcover-exact, weak_exact_of_factor_exact, norm_exact_complex}
  \leanok
  For every constant $c$,
  consider the pullback
  \[
    \begin{tikzcd}
      & M_c \rar & M_{kc} \\
      & X_c \uar \rar & M^N_{kc/N} \uar \\
      M^N_{c/N} \ar{uur} \ar{rru} \ar[dotted]{ur}
    \end{tikzcd}
  \]
  We therefore get morphisms of cochain complexes
  \[
    \begin{tikzcd}
      \hat V(M_{kc}) \rar\dar & \hat V(M_c) \rar\dar & \hat V(M_c) \dar \\
      \hat V(M^N_{kc/N}) \rar\dar & \hat V(X_c) \rar\dar & \hat V(M^N_{c/N}) \dar \\
      \vdots \rar & \vdots \rar & \vdots
    \end{tikzcd}
  \]
  where all the columns are of the form
  ``alternating face map complex of $\hat V(\_)$ applied to a Cech nerve''.
  Note that the horizontal maps are norm-nonincreasing
  and their compositions are restriction maps.

  Claim: for $c \ge c_0$, the map $X_c \to M_c$ is surjective.

  Indeed, by assumption every $x \in M_c$ can be decomposed into a sum
  $x = x_1 + \dots + x_N$ with $x_i \in M_{c/N+d} \subset M_{kc/N}$,
  since $c \ge c_0$ and $d \le \frac{(k-1)c_0}{N}$.

  By Proposition~\ref{cechcover-exact},
  the middle column is normed exact (in the sence of Definition~\ref{norm_exact_complex}).
  The result follows from Lemma~\ref{weak_exact_of_factor_exact}.
\end{proof}

\begin{proposition}
  \label{col_exact}
  \lean{thm95.col_exact}
  \leanok
  Let $d$ be the constant from Proposition~\ref{Mbar-splittable}.
  Let $k > 1$ and $c_0 > 0$ be real numbers such that
  \[
    (k - 1) * c_0 / N \ge d.
  \]
  Let $m$ be any natural number, and put
  \[
    K = (m + 2) + \frac{r + 1}{r(1 - r)} (m + 2)^2
  \]
  Finally, let $c_0'$ be $\frac{c_0}{r' \cdot n_i}$,
  where $n_i$ is the $i$-th index in our fixed Breen--Deligne data.

  Then $i$-th column in the double complex
  is $(k^2, K)$-weak bounded exact in degrees $\le m$ for $c \ge c_0'$.
\end{proposition}

\begin{proof}\leanok
  \uses{combi,lem:Tinv,canonical_iso_2,weakdualsnakelemma, pre_col_exact}
  Let $M^{(m)}$ denote $\Hom(\Lambda^{(m)},\overline{\mathcal M}_{r'}(S))^{n_i}$.
  We also write $M$ for $M^{(0)} = \Hom(\Lambda,\overline{\mathcal M}_{r'}(S))^{n_i}$
  and $M'$ for $M^{(1)}$.
  By Proposition~\ref{combi}, $M$ is $N$-splittable with error term~$d$.

  Consider the diagram of morphisms of systems of complexes
  \[
    \begin{tikzcd}
      \hat V(M_c)^{T^{-1}} \rar\dar & \hat V(M_c) \rar["{T^{-1} - [T^{-1}]^*}"]\dar & \hat V(M_c) \dar \\
      \hat V(M'_c)^{T^{-1}} \rar\dar & \hat V(M'_c) \rar["{T^{-1} - [T^{-1}]^*}"]\dar & \hat V(M'_c) \dar \\
      \vdots \dar & \vdots \dar & \vdots \dar \\
      \hat V(M^{(m)}_c)^{T^{-1}} \rar & \hat V(M^{(m)}_c) \rar["{T^{-1} - [T^{-1}]^*}"] & \hat V(M^{(m)}_c)
    \end{tikzcd}
  \]

  By Lemmas~\ref{pre_col_exact} and~\ref{canonical_iso_2},
  we know that the two columns on the right are
  weakly $\le k$-exact in degrees $\le m$ and for $c \ge c_0$ with bound~$1$.

  The result now follows from Lemma~\ref{lem:Tinv}, and Proposition~\ref{weakdualsnakelemma}.
\end{proof}

\begin{proposition}
  \label{double-complex-htpy}
  \lean{NSC_htpy}
  \uses{double_complex, BD_h_mul_suitable, spectral-htpy}
  \leanok
  Let $h$ be the homotopy packaged with $\BD$,
  and let $h^N$ denote the $n$-th iterated composition of $h$
  (see Def~\ref{BD_h_mul})
  which is a homotopy between
  $\pi^N$ and $\sigma^N \colon N \otimes \BD \to \BD$.

  Let $H \in \R_{\ge 0}$ be such that for $i = 0, \dots, m$
  the universal map $h^N_i$ is bound by $H$
  (see Def~\ref{universal_map_bound_by}).

  Then the double complex satisfies
  the normed homotopy homotopy condition (Def~\ref{spectral-htpy})
  for $m$, $H$, and $c_0$.
\end{proposition}

\begin{proof}
  \uses{exists_adept, BD_h_mul_suitable, row_one_iso}
  \leanok
  By Lemma~\ref{row_one_iso} we may replace the first row by
  \[
    C^{N \otimes \BD}_{\kappa/N}(\Hom(\Lambda, M))^\bullet.
  \]
  Now it is important to recall that we have chosen $k' \ge \kappa'_i$ for all $i = 0, \dots, m+1$.

  Our goal is to find,
  in degrees $\leq m$, a homotopy between the two maps from the first row
  \[
    \widehat{V}(\Hom(\Lambda,M)_{\leq c})^{T^{-1}}\to \widehat{V}(\Hom(\Lambda,M)_{\leq \kappa_1c}^2)^{T^{-1}}\to \ldots
  \]
  to the second row
  \[
    \widehat{V}(\Hom(\Lambda,M)_{\leq c/N}^N)^{T^{-1}}\to \widehat{V}(\Hom(\Lambda,M)_{\leq \kappa_1c/N}^{2N})^{T^{-1}}\to \ldots
  \]
  respectively induced by $\sigma^N$ and $\pi^N$ (which are maps $N \otimes \BD$

  By Definition~\ref{BD_h_mul} and Lemma~\ref{BD_h_mul_suitable}
  we can find this homotopy between the complex for $k'c$ and the complex for~$c$.
  (Here we use $k'\geq c_i'$ for $i=0,\ldots,m$.)
  By assumption, the norm of these maps is bounded by~$H$.

  Finally, it remains to establish the estimate (eq.~\ref{eq:homotopicmapsmall}) on the homotopic map.
  We note that this takes $x\in \widehat{V}(\Hom(\Lambda,M)_{\leq k'^2\kappa_ic}^{a_i})^{T^{-1}}$
  (with $i=q$ in the notation of (eq.~\ref{eq:homotopicmapsmall})) to the element
  \[
    y\in \widehat{V}(\Hom(\Lambda,M)_{\leq k'\kappa_ic/N}^{Na_i})^{T^{-1}}
  \]
  that is the sum of the $N$ pullbacks along the $N$ projection maps
  $\Hom(\Lambda,M)_{\leq k'\kappa_ic/N}^{Na_i}\to \Hom(\Lambda,M)_{\leq k'^2\kappa_ic}^{a_i}$.
  We note that these actually take image in $\Hom(\Lambda,M)_{\leq \kappa_ic}^{a_i}$ as $N\geq k'$,
  so this actually gives a well-defined map
  \[
    \widehat{V}(\Hom(\Lambda,M)_{\leq \kappa_ic}^{a_i})^{T^{-1}} \to
    \widehat{V}(\Hom(\Lambda,M)_{\leq k'\kappa_ic/N}^{Na_i})^{T^{-1}}.
  \]
  We need to see that this map is of norm $\leq \epsilon$.
  Now note that by our choice of $N$, we actually have $k'\kappa_ic/N\leq (r')^b \kappa_ic$,
  so this can be written as the composite of the restriction map
  \[
    \widehat{V}(\Hom(\Lambda,M)_{\leq \kappa_ic}^{a_i})^{T^{-1}} \to
    \widehat{V}(\Hom(\Lambda,M)_{\leq (r')^b \kappa_ic}^{a_i})^{T^{-1}}
  \]
  and
  \[
    \widehat{V}(\Hom(\Lambda,M)_{\leq (r')^b \kappa_ic}^{a_i})^{T^{-1}} \to
    \widehat{V}(\Hom(\Lambda,M)_{\leq k'\kappa_ic/N}^{Na_i})^{T^{-1}}.
  \]
  The first map has norm exactly $r^b$, by $T^{-1}$-invariance,
  and as multiplication by $T$ scales the norm with a factor of $r$ on $\widehat{V}$.
  (Here is where we use $r'>r$, ensuring different scaling behaviour of the norm on source and target.)
  The second map has norm at most $N$ (as it is a sum of $N$ maps of norm $\leq 1$).
  Thus, the total map has norm $\leq r^bN$. But by our choice of $N$, we have $r^bN\leq \epsilon$, giving the result.
\end{proof}

\begin{proof}[Proof of Theorem~\ref{explicit}]
  \proves{explicit}
  \leanok
  \uses{spectral,col_exact, double-complex-htpy}
  By induction, the first condition of Proposition~\ref{spectral}
  is satisfied for all $c\geq c_0$ with $c_0$ large enough
  (depending on $\Lambda$ but not $V$ or $S$).

  The second condition is Proposition~\ref{col_exact},
  and the third condition has been checked in Proposition~\ref{double-complex-htpy}.

  Thus, we can apply Proposition~\ref{spectral},
  and get the desired $\leq \max(k'^2,2k_0H)$-exactness in degrees $\leq m$ for $c\geq c_0$,
  where $k'$, $k_0$ and $H$ were defined only in terms of $k$, $m$, $r'$ and $r$,
  while $c_0$ depends on $\Lambda$ (but not on $V$ or $S$).
  This proves the inductive step.
\end{proof}

\begin{question}
  Can one make the constants explicit, and how large are they?
  \footnote{A back of the envelope calculation seems to suggest that $k$ is roughly doubly exponential in $m$, and that $N$ has to be taken of roughly the same magnitude.}
  Modulo the Breen-Deligne resolution, all the arguments give in principle explicit constants;
  and actually the proof of the existence of the Breen-Deligne resolution
  should be explicit enough to ensure the existence of bounds on the $c_i$ and $c_i'$.

  Answer: yes, we can do this.
  And we should write up a clean account asap,
  when we have cleaned up the proof in Lean.
\end{question}

% \section{Relevant material that should move to a better place}

% The next statement uses the definition of the completion of a condensed abelian group, see Definition~\ref{CLC}.
%
% \begin{proposition}
%   \label{prop:normedcompletion}
%   \uses{CLC}
% The condensed abelian group $\widehat{M}$ is canonically identified with the condensed abelian group associated to the topological abelian group $\widehat{M}_{\mathrm{top}}$ given by the completion of $M$ equipped with the topology induced by the norm. The norm defines a natural map of condensed sets
% \[
% ‖\cdot‖: \widehat{M}\to \mathbb R_{\geq 0}.
% \]
% \end{proposition}
%
% \begin{proof}
% Note that in the supremum norm any continuous function from $S$ to $\widehat{M}_{\mathrm{top}}$ can be approximated by locally constant functions arbitrarily well, and that the space of continuous functions from $S$ to $\widehat{M}_{\mathrm{top}}$ is complete with respect to the supremum norm. That $‖\cdot‖$ defines a map of condensed sets $\widehat{M}\to \mathbb R_{\geq 0}$ follows for example from this identification with $\underline{\widehat{M}_{\mathrm{top}}}$, as the norm is by definition a continuous map $\widehat{M}_{\mathrm{top}}\to \mathbb R_{\geq 0}$.
% \end{proof}


% vim: ts=2 et sw=2 sts=2
