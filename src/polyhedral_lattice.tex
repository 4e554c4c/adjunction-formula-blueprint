\section{Polyhedral lattices}
\label{sec:polyhedral_lattice}

\begin{definition}
  \label{polyhedral_lattice}
  \lean{polyhedral_lattice}
  \leanok
  A \emph{polyhedral lattice} is a finite free abelian group~$\Lambda$
  equipped with a norm $‖\cdot‖_\Lambda \colon \Lambda\otimes \mathbb R\to \mathbb R$
  such that there exists a finite set $\{\lambda_1, \dots, \lambda_n\} \subset \Lambda$
  that generate the norm:
  that is to say, for every $\lambda \in \Lambda$ there exist
  $c_1, \dots, c_n \in \mathbb Q$ such that
  $\lambda = \sum c_i \lambda_i$ and $\|\lambda\| = c_i\|\lambda_i\|$.

  Equivalently (but not verified in Lean):
  the norm is given by the supremum of finitely many linear functions on $\Lambda$;
  or once more,
  equivalently, the ``unit ball''
  $\{\lambda\in \Lambda\otimes \mathbb R\mid ‖\lambda‖_\Lambda\leq 1\}$ is a polyhedron.
\end{definition}

Finally, we can prove the key combinatorial lemma,
ensuring that any element of $\Hom(\Lambda,\overline{\mathcal M}_{r'}(S))$
can be decomposed into $N$ elements whose norm is roughly $\tfrac 1N$ of the original element.

\begin{definition}
  \label{splittable}
  \uses{pseudo_normed_group}
  \lean{pseudo_normed_group.splittable}
  \leanok
  Let $M$ be a pseudo-normed group, $N \in \mathbb N$, and $d \in \mathbb R_{\ge 0}$.
  We say that $M$ is \emph{$N$-splittable} with error term~$d$,
  if for all $c$ and $x \in M_c$,
  there exists a decomposition
  \[
    x = x_1 + x_2 + \dots + x_N,
  \]
  with $x_i \in M_{c/N + d}$.
\end{definition}

\begin{lemma}
  \label{combi}
  \lean{lem98}
  \leanok
  \uses{polyhedral_lattice}
  Let $\Lambda$ be a polyhedral lattice.
  Then for all positive integers $N$ there is a constant $d$
  such that for all $c>0$ one can write any
  $x\in \Hom(\Lambda,\overline{\mathcal M}_{r'}(S))_{\leq c}$ as
  \[
    x=x_1+\ldots+x_N
  \]
  where all $x_i\in \Hom(\Lambda,\overline{\mathcal M}_{r'}(S))_{\leq c/N+d}$.

  In other words, for all $N$, there exists a $d$ such that
  $\Hom(\Lambda, \overline{\mathcal M}_{r'}(S))$ is $N$-splittable with error term~$d$.
\end{lemma}

As preparation for the proof, we have the following results.

\begin{lemma}[Gordan's lemma]
  \label{explicit_gordan}
  \lean{explicit_gordan}
  \leanok
  Let $\Lambda$ be a finite free abelian group,
  and let $\lambda_1, \ldots, \lambda_m \in \Lambda$ be elements.
  Let $M \subset \Hom(\Lambda, \mathbb Z)$ be the submonoid
  \(\{x \mid x(\lambda_i) \ge 0 \text{ for all \(i = 1, \dots, m\)}\}\).
  Then $M$ is finitely generated as monoid.
\end{lemma}

\begin{proof}
  \leanok
  This is a standard result. We omit the proof here. It is done in Lean.
\end{proof}

\begin{lemma}
  \label{combi_aux}
  \lean{lem97}
  \leanok
  Let $\Lambda$ be a finite free abelian group,
  let $N$ be a positive integer,
  and let $\lambda_1,\ldots,\lambda_m\in \Lambda$ be elements.
  Then there is a finite subset $A\subset \Lambda^\vee$
  such that for all $x\in \Lambda^\vee=\Hom(\Lambda,\mathbb Z)$
  there is some $x'\in A$ such that $x-x'\in N\Lambda^\vee$
  and for all $i=1,\ldots,m$,
  the numbers $x'(\lambda_i)$ and $(x-x')(\lambda_i)$ have the same sign,
  i.e.~are both nonnegative or both nonpositive.
\end{lemma}

\begin{proof}
  \uses{explicit_gordan}
  \leanok
  It suffices to prove the statement for all $x$ such that $\lambda_i(x)\geq 0$ for all $i$;
  indeed, applying this variant to all $\pm \lambda_i$, one gets the full statement.

  Thus, consider the submonoid $\Lambda^\vee_+\subset \Lambda^\vee$
  of all $x$ that pair nonnegatively with all $\lambda_i$.
  This is a finitely generated monoid by Lemma~\ref{explicit_gordan};
  let $y_1,\ldots,y_M$ be a set of generators.
  Then we can take for $A$ all sums $n_1y_1+\ldots+n_My_M$ where all $n_j\in \{0,\ldots,N-1\}$.
\end{proof}

\begin{lemma}
  \label{exists_partition}
  \lean{combinatorial_lemma.exists_partition}
  \leanok
  Let $x_0, x_1, \dots$ be a sequence of reals,
  and assume that $\sum_{i=0}^\infty x_i$ converges absolutely.
  For every natural number $N > 0$,
  there exists a partition $\mathbb N = A_1 \sqcup A_2 \sqcup \dots \sqcup A_N$
  such that for each $j = 1,\dots,N$ we have
  $\sum_{i \in A_j} x_i \le (\sum_{i=0}^\infty x_i)/N + 1$
\end{lemma}

\begin{proof}
  \leanok
  Define the $A_j$ recursively:
  assume that the natural numbers $0, \dots, n$
  have been placed into the sets $A_1, \dots, A_N$.
  Then add the number $n+1$ to the set $A_j$ for which
  \[
    \sum_{i=0, i\in A_j}^n x_i
  \]
  is minimal.
\end{proof}

\begin{lemma}
  \label{lem98_int}
  \lean{lem98_int}
  \leanok
  For all natural numbers $N > 0$,
  and for all $x\in \overline{\mathcal M}_{r'}(S)_{\leq c}$
  one can decompose $x$ as a sum
  \[
    x=x_1+\ldots+x_N
  \]
  with all $x_i\in \overline{\mathcal M}_{r'}(S)_{\leq c/N+1}$.
\end{lemma}

\begin{proof}
  \leanok
  \uses{exists_partition}
  Choose a bijection $S \times \mathbb N \cong \mathbb N$,
  and transport the result from Lemma~\ref{exists_partition}.
\end{proof}

\begin{proof}[{Proof of Lemma~\ref{combi}}]
  \proves{combi}
  \uses{combi_aux, lem98_int}
  \leanok
  Pick $\lambda_1,\ldots,\lambda_m\in \Lambda$ generating the norm. We fix a finite subset $A\subset \Lambda^\vee$ satisfying the conclusion of the previous lemma. Write
  \[
  x=\sum_{n\geq 1, s\in S} x_{n,s} T^n [s]
  \]
  with $x_{n,s}\in \Lambda^\vee$. Then we can decompose
  \[
  x_{n,s} = N x_{n,s}^0 + x_{n,s}^1
  \]
  where $x_{n,s}^1\in A$ and we have the same-sign property of the last lemma. Letting $x^0 = \sum_{n\geq 1, s\in S} x_{n,s}^0 T^n [s]$, we get a decomposition
  \[
  x = Nx^0 + \sum_{a\in A} a x_a
  \]
  with $x_a\in \overline{\mathcal M}_{r'}(S)$ (with the property that in the
  basis given by the $T^n [s]$, all coefficients are $0$ or $1$). Crucially,
  we know that for all $i=1,\ldots,m$, we have
  \[
  ‖x(\lambda_i)‖ = N ‖x^0(\lambda_i)‖ + \sum_{a\in A} |a(\lambda_i)| ‖x_a‖
  \]
  by using the same sign property of the decomposition.

  Using this decomposition of $x$, we decompose each term into $N$ summands.
  This is trivial for the first term $Nx^0$,
  and each summand of the second term decomposes with $d = 1$ by Lemma~\ref{lem98_int}.
  (It follows that in general one can take for $d$
  the supremum over all $i$ of $\sum_{a\in A} |a(\lambda_i)|$.)
\end{proof}

\begin{definition}
  \label{rescaled-sum}
  \uses{polyhedral_lattice}
  \leanok
  \lean{rescale.polyhedral_lattice}
  \lean{finsupp.polyhedral_lattice}
  Let $\Lambda$ be a polyhedral lattice, and let $N > 0$ be a natural number.
  (We think of $N$ as being fixed once and for all,
  and thus it does not show up in the notation below.)

  By $\Lambda'$ we denote $\Lambda^N$ endowed with the norm
  \[
	  ‖(\lambda_1,\ldots,\lambda_N)‖_{\Lambda'} = \tfrac 1N(‖\lambda_1‖_\Lambda+\ldots+‖\lambda_N‖_\Lambda).
  \]
  This is a polyhedral lattice.
\end{definition}

\begin{lemma}
  \label{polyhedral-quotient}
  \uses{rescaled-sum}
  \lean{polyhedral_lattice.conerve.obj.polyhedral_lattice}
  \leanok
  For any $m\geq 1$, let $\Lambda'^{(m)}$ be given by $\Lambda'^m / \Lambda\otimes (\mathbb Z^m)_{\sum=0}$;
  for $m=0$, we set $\Lambda'^{(0)} = \Lambda$.
  Then $\Lambda'^{(m)}$ is a polyhedral lattice.
\end{lemma}

\begin{proof}
  \leanok
  The proof is done in Lean.
  TODO: write down a proof here.
\end{proof}

\begin{definition}
  \label{cosimplicial-lattice}
  \uses{polyhedral-quotient}
  \lean{PolyhedralLattice.cosimplicial}
  \leanok
  For any $m\geq 1$, let $\Lambda'^{(m)}$ be given by $\Lambda'^m / \Lambda\otimes (\mathbb Z^m)_{\sum=0}$;
  for $m=0$, we set $\Lambda'^{(0)} = \Lambda$.
  Then $\Lambda'^{(\bullet)}$ is a cosimplicial polyhedral lattice,
  the \v{C}ech conerve of $\Lambda\to \Lambda'$.

  In particular, $\Lambda'^{(0)} = \Lambda \to \Lambda' = \Lambda'^{(1)}$
  is the diagonal embedding.
\end{definition}

\begin{definition}
  \label{Hom}
  \uses{polyhedral_lattice, profinitely_filtered_pseudo_normed_group_with_Tinv}
  \lean{polyhedral_lattice.add_monoid_hom.profinitely_filtered_pseudo_normed_group_with_Tinv}
  \leanok
  Let $\Lambda$ be a polyhedral lattice,
  and $M$ a profinitely filtered pseudo-normed group.

  Endow $\Hom(\Lambda, M)$ with the subspaces
  \[
    \Hom(\Lambda, M)_{\leq c} =
    \{f \colon \Lambda \to M \mid
      \forall x \in \Lambda, f(x) \in M_{\leq c‖x‖} \}.
  \]
  As $\Lambda$ is polyhedral, it is enough to check the given condition on~$f$
  for a finite collection of $x$ that generate the norm.

  These subspaces are profinite subspaces of $M^\Lambda$,
  and thus they make $\Hom(\Lambda, M)$ ito a profinitely filtered pseudo-normed group.

  If $M$ has an action of $T^{-1}$, then so does $\Hom(\Lambda, M)$.
\end{definition}

% vim: ts=2 et sw=2 sts=2

