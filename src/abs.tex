We develop some 
homological and element-wise machinery
from Commutative Algebra, culminating 
in the proof of the Auslander-Buchsbaum-Serre 
theorem, and 
the corollary that the condition of being 
a regular ring localizes. 
This corollary is crucial for our proof of adjunction.

We follow Bruns-Herzog almost exclusively.


%TODO: Bring all Nakayama corollaries 
%      here into section one, prove them,
%      and then add references to the correct
%      corollaries anywhere that 'nakayama' or 'Nakayama'
%      appears in this document.

\subsection{Projective Resolutions}

\begin{definition}
  \label{def:projv_resl}
  Let $R$ be a ring. 
  Let $M$ be an $R$-module.
  A projective resolution of $M$ over $R$
  Is an exact sequence
  \[
    \ldots \to P^2 \to P^1 \to P^0 \to M \to 0
  .\]
  That is to say, a projective resolution is a
  quasi-isomorphism of complexes between the 
  inclusion of $M$ into chain complexes over $R$
  (i.e. the complex which has $M$ in degree zero,
  trivial everywhere else, with trivial maps)
  and a bounded—below complex whose
  components are projective $R$-modules.
\end{definition}

\begin{definition}
  \label{def:free_resl}
  \uses{def:projv_resl}
  A free resolution of $M$ over $R$ is 
  a projective resolution whose components
  $P^i$ are free.
\end{definition}

\begin{definition}
  \label{def:length_of_resl}
  \uses{def:projv_resl}
  The length of a projective
  resolution $P_\cdot$ is 
  the highest $i$ such that $P^i$
  is nonzero.
  If there exists no such $i$, the
  length is infinity
\end{definition}

\begin{definition}
  \label{def:projdim}
  \uses{def:length_of_resl}
  Let $M$ be an $R$-module. 
  Then $\projdim M$ is the minimum 
  length of a projective resolution of $M$.
  It lives in the set $\mathbb{N} \cap \infty$.
\end{definition}

\begin{lemma}
  \label{lem:finite_resl_implies_finite_projdim}
  \uses{def:projdim}
  If there exists a projective resolution of
  $M$ with finite length, then 
  $\projdim M < \infty$.
\end{lemma}

\begin{proof}
  Expand definitions, use the definition of minimum.
\end{proof}

\begin{lemma}
  \label{lem:min_resl_has_min_length}
  Any resolution of $M$ over $R$ has 
  length at least that of the minimal resolution.
\end{lemma}

\begin{proof}
  Expand definitions, use definition of minimum
\end{proof}

\begin{proposition}
  \label{prop:existence_min_free_resl}
  \uses{def:free_resl,def:projdim}
  There exists a free resolution $F_\cdot$
  of $M$ such that length of $F_\cdot$
  is equal to the projective dimension of $M$.
\end{proposition}

\begin{proof}
  One has the generators-relations explicit construction,
  which should eventually get its own definition.
  Then one has to show that this is indeed minimal among
  projective resolutions, which I don't remember how to do
  the top of my head.
\end{proof}

%\begin{lemma}
%  \label{lem:projdim_len_minimal_free}
%  Let $R$ be a local noetherian ring, 
%  and $M$ be an $R$-module.
%  Then $\projdim M$ is equal to the length
%  of a minimal free resolution of $M$.
%\end{lemma}

\begin{definition}
  \label{def:globdim}
  \uses{def:projdim}
  The global dimension of a ring $R$
  is the supremum over all $R$-modules
  of $\projdim M$.
\end{definition}

% MARK - localization and resolutions

\begin{lemma}
  \label{lem:loczn_exact}
  Localization by a multiplicative set is an exact functor.
\end{lemma}

\begin{lemma}
  \label{lem:loczn_of_resl_is_resl}
  \uses{lem:loczn_exact}
  Let $M$ be an $R$-module.
  Let $S$ be a multiplicative set.
  Let $\ldots \to A_2 \to A_1 \to A_0 \to M \to 0$
  be a free resoltion of $M$.
  Then
  $\ldots S^{-1}(A_2) \to S^{-1}(A_1) \to S^{-1}(A_0) \to S^{-1}M \to 0$
  is a free resolution of $S^{-1}M$.
\end{lemma}


% TODO: figure out what we "really want to prove" with localizations and resolutions
%         this is kind of a cop-out

\subsection{Dimension Theory}

\begin{definition}
  \label{def:dim_module}
  The krull dimension of a module 
  is the supremum of the lengths
  of chains of prime submodules.
\end{definition}

\begin{definition}
  \label{def:ring}
  Krull dim of ring (prime ideals
  are prime submodules, might need 
  to show this)
\end{definition}

Other dimension theory necessary? Probably?

\subsection{Regular elements and sequences}

\begin{definition}
  \label{def:reg_elt}
  Let $M$ be an $R$-module.
  An element $x \in R$ is $M$-regular if it does 
  not annihilate any element in $M$.
  In colon notation, it is non a member of the ideal
  $( 0 : M )$.
  If $x$ is an $R$-regular element, we simply say it is 
  a regular element.
\end{definition}

\begin{definition}
  \label{def:weak_reg_seq}
  \uses{def:reg_elt}
  Let $M$ be an $R$-module.
  Then a weak $M$-regular sequence
  is a sequence $x_1, \ldots, x_n$ 
  with $x_i \in R$ for all $i$,
  such that 
  $x_i$ is a
  $M / (x_1, \ldots, x_{i-1}) M$-regular
  element for all $i$.
  If $M = R$, we say that $x_1, \ldots, x_n$
  is a weak regular sequence. 
\end{definition}

\begin{definition}
  \label{def:reg_seq}
  \uses{def:weak_reg_seq}
  Let $M$ be an $R$-module.
  An $M$-regular sequence is a weak $M$-regular sequence
  such that $M / (x_1, \ldots, x_n) M \neq 0$.
  If $M = R$, we simply call this a regular sequence.
\end{definition}

\begin{theorem}
  \label{thm:reg_seq_part_of_sys_param}
  \uses{def:reg_seq}
  Any regular sequence is part of a system of parameters
\end{theorem}

\begin{proof}
  This depends on the notion of depth, a bunch of machinery
  from BH chapter 1, and the definition of associated primes.
\end{proof}

\begin{definition}
  \label{def:reg_loc}
  A local ring $R$ is regular if 
  the minimal number of generators of its
  maximal ideal is equal to the dimension of $R$.
\end{definition}

\begin{definition}
  \label{def:reg_ring}
  \uses{def:reg_loc}
  A ring $R$ is regular if 
  $R_\mathfrak{p}$ is regular
  for every $\mathfrak{p} \in R$.
\end{definition}

\begin{proposition}
  \label{prop:reg_def_equiv}
  \uses{def:reg_loc}
  The following are equivalent
  Firstly, $R$ is regular.
  Secondly, the zariski cotangent space
  is a vector space of dimension $\dim R$.
\end{proposition}

\begin{proof}
  See in adjunction_blob.txt
\end{proof}

\begin{lemma}
  \label{lem:reg_int_dom}
  \uses{def:reg_loc}
  Every regular ring is an integral domain.
\end{lemma}

\begin{proposition}[BH 2.2.4]
  \label{prop:reg_quot_sys_param}
  Let $R$ be a regular local ring. 
  Then $R / I$ is regular local if and only if 
  $I$ is generated by a (regular) system of parameters
  (I.e. a generating set for $\mathfrak{m}$).
\end{proposition}

\begin{proof}
  \uses{lem:reg_int_dom}
  This proof uses the following facts:

  * a Nakayama corollary
  
  * the fact that regular rings are integral domains

  * the fact that you can't have a proper containment
    of integral domains with the same dimension
\end{proof}

\begin{proposition}
  \label{prop:reg_loc_maximal_reg_seq}
  \uses{def:reg_loc,def:reg_seq}
  A local ring $R$ is regular if and only if
  its maximal ideal is generated by a regular sequence.
\end{proposition}

\begin{proof}
  \uses{lem:reg_int_dom,thm:reg_seq_part_of_sys_param}
  See Bruns-Herzog 2.2.5. This uses BH 2.2.4 for the forward
  direction,
  and for the reverse direction, we use BH 1.2.12 and the fact that
  the minimal number of generators for $\mathfrak{m}$ is at least
  $\dim R$ (need to account for the word ``system of parameters''.
\end{proof}

\subsection{Associated Primes}

\begin{definition}
  \label{def:support_module}
  The support of a module is the set of prime ideals
  $\mathfrak{p} \ins R$ such that 
  $M_\mathfrak{p} \neq 0$.
\end{definition}

\begin{proposition}
  \label{prop:assoc_primes_def_equiv}
  Let $R$ a ring, and $M$ an $R$-module.
  Let $\mathfrak{p}$ be a prime ideal, 
  then 
  the following are equivalent:
  \begin{enumerate}[(i)]
    \item $\mathfrak{p} = \Ann_R (m)$ for some element 
      $m \in M$.
    \item $R / \mathfrak{p}$ embeds into $M$.
      % \item something about prime submodules
  \end{enumerate}
\end{proposition}

\begin{definition}
  \label{def:assoc_primes}
  \uses{prop:assoc_primes_def_equiv}
  The set $\Assoc_R (M)$ is the set of
  primes satisfying the preivious proposition
\end{definition}

\begin{proposition}
  \label{prop:assoc_primes_in_support}
  The associated primes of $M$ are in the
  support of $M$.
\end{proposition}

\subsection{Depth and Regular Sequences}

This mostly follows Bruns-Herzog 1.2

\begin{definition}
  \label{def:depth}
  The depth of a module $M$ is . . . 
\end{definition}


\subsection{Auslander-Buchsbaum Formula}

\begin{lemma}
  \label{lem:module_assoc_maximal_tensor_free_summand}
  BH 1.3.4
\end{lemma}

\begin{proof}
  \uses{prop:assoc_primes_def_equiv}
  Uses

  * def of associated primes

  * a fact about associated primes giving and embedding

  * some facts about commutative squares (maybe already in lean?)

  * tensor products of modules

  * Nakayama (a corollary, like Atiyah-MacDonald 2.8 
    but uses maps, might just follow from AM 2.8)
\end{proof}

\begin{lemma}
  \label{lem:projdim_preserved_quotient_reg_elt}
  BH 1.3.5
\end{lemma}

\begin{proof}
  This uses 

  * BH 1.1.5

  * The "tor" characterization of projective dimension

\end{proof}

\begin{theorem}
  \label{thm:auslander_buchsbaum_formula}
  depth + projdim = depth
\end{theorem}

\begin{proof}
  \uses{
    lem:module_assoc_maximal_tensor_free_summand,
    lem:projdim_preserved_quotient_reg_elt,
  }
\end{proof}


%TODO: figure out if we will actually use this.

%\subsection{Cohen-Macaulay Rings}
%
%
%Let $(R,\mathfrak{m},k)$ be a noetherian local ring.
%
%\begin{definition}
%  \label{def:local_cm}
%  \uses{def:depth}
%  An $R$-module $M$ is 
%  Cohen-Macaulay if $\depth M = \dim M$.
%  We say that $R$ is a Cohen-Macaulay ring if
%  $R$ is Cohen-Macaulay as a module.
%\end{definition}
%
%\begin{definition}
%  \label{def:maxml_local_cm}
%  \uses{def:dim_module}
%  $M$ is maximal Cohen-Macaulay if 
%  it has $\dim M = \dim R$.
%\end{definition}
%
%Now, let $R$ be an arbitrary noetherian ring.
%
%\begin{definition}
%  \label{def:cm}
%  \uses{def:support_module,def:local_cm}
%  $M$ is Cohen-Macaulay if 
%  $M_\mathfrak{m}$ is for all maximal
%  ideals $\mathfrak{m}$ in the support of $M$.
%\end{definition}
%
%\begin{corollary}
%  \label{cor:zero_is_cm}
%  \uses{def:cm}
%  The zero module is Cohen-Macaulay.
%\end{corollary}
%
%\begin{definition}
%  \label{def:maxml_cm}
%  \uses{def:maxml_local_cm}
%  $M$ is maximal Cohen-Macaulay if 
%  if $M_\mathfrak{m}$ is for all 
%  maximal ideals $\mathfrak{m}$.
%  I.e. there is no support condition as before.
%\end{definition}
%
%\begin{theorem}
%  [BH 2.1.2]
%  \label{thm:cm_reg_seq_iff_sys_param}
%  Let $M$ be a Cohen-Macaulay $R$-module.
%  Then $x_1, \ldots, x_n$ is a regular sequence
%  if and only if it is part of a system of parameters.
%\end{theorem}
%
%\begin{proof}
%  \uses{
%    prop:assoc_primes_in_support,
%  }
%  Uses
%  
%  * BH 1.2.13
%
%  * the fact that the associated primes are in the support
%
%  * some stuff about depth and grade and dimension
%
%  * 1.6.19
%
%  OR, use hochster's proof in
%  http://www.math.lsa.umich.edu/~hochster/615W14/CM.pdf
%\end{proof}
%
%\begin{proposition}
%  \label{prop:cm_syzygy}
%  Let $R$ be Cohen-Macaulay.
%  Then any dth syzygy module is a maximal cohen-macaulay or zero
%
%\end{proposition}
%
%\begin{proof}
%  https://math.stackexchange.com/questions/159390/bruns-herzog-problem-2-1-26-page-64
%\end{proof}

\subsection{Proof of Auslander-Buchbaum-Serre}



\begin{lemma}
  \label{lem:reg_implies_finite_global}
  \uses{def:globdim}
  Let $R$ be a regular local ring.
  Then $R$ has finite global dimension.
  That is, any finitely generated module $R$ has finite
  projective dimension.
\end{lemma}

\begin{proof}
  %TODO
\begin{proof}

\begin{theorem}
  [Ferrand-Vasconcelos, BH 2.2.8]
  \label{thm:ferrand_vasconcelos}
  \uses{def:reg_seq,def:projdim}

  Let $(R,\mathfrak{m},k)$ be a local noetherian ring.
  Let $I$ be a nonzero ideal with finite projective dimension.
  If $I / I^2$ is a free $R$-module, 
  then $I$ is generated by a regular sequence.
\end{theorem}

\begin{proof}
  Since $I$ has finite projective dimension, it has a finite free
  resolution.
  Thus, by 1.4.6 it has must have an $R$-regular element $x$.
  . . . finish the proof . . . 
\begin{proof}

% MARK: Auslander-Buchsbaum-Serre (the goal)

\begin{theorem}[Auslander-Buchsbaum-Serre Criterion, BH 2.2.7]
  \label{thm:auslander_buchsbaum_serre}
  Let $(R,\mathfrak{m},k$ be a noetherian local ring.
  The following are equivalent:
  \begin{enumerate}[(i)]
    \item $R$ is regular.
    \item $R$ has finite global dimension.
    \item $\projdim k < \infty$
  \end{enumerate}
\end{theorem}

\begin{proof}
  \uses{
    lem:reg_implies_finite_global,
    thm:ferrand_vasconcelos,
    prop:reg_loc_maximal_reg_seq
  }

  (i) $\implies$ (ii) is precisely Lemma \ref{lem:reg_implies_finite_global}

  (ii) $\implies$ (iii) follows by applying the definition of global dimension with $M=k$

  (iii) $\implies$ (i) is a special case of Theorem \ref{thm:ferrand_vasconcelos},
  using \ref{prop:reg_loc_maximal_reg_seq} to 
  conclude regularity.

  \ref{thm:ferrand_vasconcelos}.
\end{proof}

\begin{theorem}[Regular Rings Localize, BH 2.2.9]

  \label{thm:regular_rings_localize}
  Let $R$ be a regular local ring, and let
  $\mathfrak{p}$ be a prime ideal in $R$.
  Then $R_\mathfrak{p}$ is a regular local ring.
\end{theorem}

% probably we want to carry around the dimension n

\begin{proof}
  \uses{thm:auslander_buchsbaum_serre,prop:existence_min_free_resl,lem:loczn_of_resl_is_resl,lem:min_resl_has_min_length}
  By Auslander-Buchsbaum-Serre, it is enough to show that 
  $R_\mathfrak{p} / \mathfrak{p}R_\mathfrak{p}$ has finite projective dimension.
  Also by Auslander-Buchsbaum-Serre, we know that 
  $k = R / \mathfrak{m}$ has finite projective dimension.
  Then $k$ has a minimal free resolution of finite length by
  Proposition \ref{existence_min_free_resl}.
  By the fact that the loclization of a resolution is a resolution,
  We get a finite resolution for 
  $R_\mathfrak{p} / \mathfrak{p}R_\mathfrak{p}$,
  thus any minimal resolution is also finite, giving us what we want.
\end{proof}
