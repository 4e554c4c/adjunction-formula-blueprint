\subsection{More Breen--Deligne}%
\label{sec:more_breen_deligne}


We need the following results about the Breen-Deligne resolution for normed abelian groups. Let us consider here abelian groups $M$ (in any topos) equipped with an increasing filtration $M_{\leq c}\subset M$ by subobjects indexed by the positive real numbers, such that $0\in M_{\leq c}$, $-M_{\leq c} = M_{\leq c}$ and $M_{\leq c}+M_{\leq c'}\subset M_{\leq c+c'}$; we need no further conditions. Let us call these pseudo-normed abelian groups.

Fix a choice of a functorial Breen-Deligne resolution
\[
C(M):\ldots \to \mathbb Z[M^{a_i}]\to\ldots\to\mathbb
Z[M^{a_1}]\to\mathbb Z[M^{a_0}]\to M\to 0
\]
of an abelian group $M$; purely for notational convenience, we can and do assume that each term is of the form $\mathbb Z[M^{a_i}]$ (as opposed to a finite direct sum of such). The possibility of doing this follows from the proof of \cite[Theorem 4.10]{Condensed}, noting that a functor of the form $A\mapsto \mathbb Z[A^n]\oplus \mathbb Z[A^m]$ admits a surjection from the functor $A\mapsto \mathbb Z[A^{n+m}]\oplus \mathbb Z$; this gives a resolution where all terms are of the form $\mathbb Z[A^{a_i}]\oplus \mathbb Z^m$. Now pass to the quotient of these complexes corresponding to the map $0\to A$; this gives a complex all of whose terms are of the form $\mathbb Z[A^{a_i}]/\mathbb Z$. Noting that $\mathbb Z[A^{a_i}]$ is functorially isomorphic to $\mathbb Z[A^{a_i}]/\mathbb Z\oplus \mathbb Z$ (via splitting $0\to A^{a_i}\to 0$), we can then add an acyclic complex of $\mathbb Z$'s in each degree to get a resolution all of whose terms are of the form $\mathbb Z[A^{a_i}]$.

%% \begin{lemma}\label{lem:constantsdeligne} There are universal constants $c_0=1,c_1,c_2,\ldots$ so that the Breen-Deligne resolution admits the subcomplex
%% \[
%% C(M)_{\leq c}: \ldots \to \mathbb Z[M^{a_i}_{\leq c_ic}]\to\ldots\to\mathbb
%% Z[M^{a_1}_{\leq c_1c}]\to\mathbb Z[M^{a_0}_{\leq c}]
%% \]
%% for all pseudo-normed abelian group objects in any topos as above, and all $c>0$.
%% \end{lemma}

%% \begin{proof} Each differential in the Breen-Deligne resolution is a finite sum of maps induced by maps $M^{a_{i+1}}\to M^{a_i}$ given by some $a_i\times a_{i+1}$-matrix of integers. Given $c_i$, one can thus find some $c_{i+1}$ so that $M^{a_{i+1}}_{\leq c_{i+1}c}$ maps into $M^{a_i}_{\leq c_ic}$ for each of those finitely many maps, which gives the claim.
%% \end{proof}

We also need some homotopies. More precisely, we start with the following homotopy.

\begin{lemma}
	\label{adept}
	\uses{BD_homotopy}
	\lean{breen_deligne.package.adept.htpy_suitable}
	\leanok
	For an abelian group $M$, the maps $\sigma_1,\sigma_2$ from
\[
C(M^2): \ldots \to \mathbb Z[M^{2a_i}]\to\ldots\to\mathbb Z[M^{2a_1}]\to\mathbb Z[M^{2a_0}]
\]
to
\[
C(M): \ldots \to \mathbb Z[M^{a_i}]\to\ldots\to\mathbb Z[M^{a_1}]\to\mathbb Z[M^{a_0}],
\]
induced by addition $M^2\to M$, respectively the sum of the two maps induced by two projections $M^2\to M$, are homotopic, via some functorial homotopy
\[
h_i: \mathbb Z[M^{2a_i}]\to \mathbb Z[M^{a_{i+1}}].
\]

If $M$ is a pseudo-normed abelian group object in any topos, then $\sigma_1$ and $\sigma_2$ are well-defined as maps of complexes from
\[
C(M^2)_{\leq c/2}: \ldots \to \mathbb Z[M^{2a_i}_{\leq c_ic/2}]\to\ldots\to\mathbb
Z[M^{2a_1}_{\leq c_1c/2}]\to\mathbb Z[M^{2a_0}_{\leq c/2}]
\]
to
\[
C(M)_{\leq c}: \ldots \to \mathbb Z[M^{a_i}_{\leq c_ic}]\to\ldots\to\mathbb
Z[M^{a_1}_{\leq c_1c}]\to\mathbb Z[M^{a_0}_{\leq c}]
\]
for all $c>0$. In that case, for all $i\geq 0$ there are universal constants $c_i'$ such that $h_i$ defines well-defined maps
\[
\mathbb Z[M^{2a_i}_{\leq c_ic/2}]\to \mathbb Z[M^{a_{i+1}}_{\leq
c_i'c_{i+1}c}]
\]
for all $c>0$.
\end{lemma}

\begin{proof}
	\leanok
	This is a consequence of the proof of the existence of the Breen-Deligne resolution,
	proved in the same way as \cite[Proposition 4.17]{Condensed}.
	The existence of the constants $c_i'$ is again formal, as in the last lemma.
\end{proof}

Now we need the following generalization to adding $N$ elements.

\begin{lemma}
	\label{homotopyNelements}
	\uses{adept}
	\lean{breen_deligne.package.adept.homotopy_mul_suitable}
	\leanok
	Let $N$ be a power of $2$. The maps of complexes $\sigma_1,\sigma_2$ from
\[
C(M^N): \ldots \to \mathbb Z[M^{Na_i}]\to\ldots\to\mathbb Z[M^{Na_1}]\to\mathbb Z[M^{Na_0}]
\]
to
\[
C(M): \ldots \to \mathbb Z[M^{a_i}]\to\ldots\to\mathbb Z[M^{a_1}]\to\mathbb Z[M^{a_0}],
\]
induced by addition $M^N\to M$, respectively the sum of the $N$ maps induced by the $N$ projections $M^N\to M$, are homotopic, via some functorial homotopy
\[
h_i^N: \mathbb Z[M^{Na_i}]\to \mathbb Z[M^{a_{i+1}}]
\]
which moreover satisfies the following bound, with the same constants $c_0',c_1',\ldots$ as in the previous lemma:

If $M$ is a pseudo-normed abelian group object in any topos, then $\sigma_1$ and $\sigma_2$ are well-defined as maps of complexes from
\[
C(M^N)_{\leq c/N}: \ldots \to \mathbb Z[M^{Na_i}_{\leq c_ic/N}]\to\ldots\to\mathbb Z[M^{Na_1}_{\leq c_1c/N}]\to\mathbb Z[M^{Na_0}_{\leq c/N}]
\]
to
\[
C(M)_{\leq c}: \ldots \to \mathbb Z[M^{a_i}_{\leq c_ic}]\to\ldots\to\mathbb Z[M^{a_1}_{\leq c_1c}]\to\mathbb Z[M^{a_0}_{\leq c}]
\]
for all $c>0$. In that case, $h_i^N$ defines well-defined maps
\[
\mathbb Z[M^{Na_i}_{\leq c_ic/N}]\to \mathbb Z[M^{a_{i+1}}_{\leq c_i'c_{i+1}c}]
\]
for all $c>0$.
\end{lemma}

\begin{proof}
	\leanok
	Let $N=2^m$.
	For each $j=0,\ldots,m-1$,
	the two maps from $C(M^{2^{j+1}})$ to $C(M^{2^j})$ from the previous lemma are homotopic,
	and we use the homotopy from that lemma.
	Composing homotopies (which amounts concretely to a certain sum)
	we get the desired homotopy from $C(M^{2^m})$ to $C(M)$.
	It follows directly from this construction that the constants $c_i'$ are unchanged.
\end{proof}

% vim: ts=2 et sw=2 sts=2
